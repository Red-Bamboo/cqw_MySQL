\documentclass[UTF8]{article}
\usepackage{mathtools,amsmath,graphicx,array}
% 加载中文环境
\usepackage[fontset=fandol]{ctex}
\usepackage{fancyvrb}
\usepackage{listings}
\usepackage{minted}% 语法高亮和代码样式设置方面更加强大和灵活
\usepackage{algorithm}
\usepackage{float}
\usepackage{booktabs}%提供命令\toprule、\midrule、\bottomrule
\usepackage[hidelinks]{hyperref}
\usepackage{amssymb} % 斜体支持
\usepackage{mdframed} % 创建带边框的区域
\usepackage{xcolor} % 颜色支持
% 定义橙盒子
\newmdenv[
    backgroundcolor=orange!10,
    linecolor=orange!30,
    linewidth=1pt,
    roundcorner=5pt,
    frametitlebackgroundcolor=orange!20,
    frametitlefont=\normalfont\bfseries,
    frametitlealignment=\raggedright
]{orangebox}
%定义红盒子
\newmdenv[
    backgroundcolor=red!10, % 正文背景颜色
    linecolor=red!30, % 边框线颜色
    linewidth=1pt, % 边框线的粗细
    roundcorner=5pt, % 圆角半径
    frametitlebackgroundcolor=red!20, % 标题背景颜色
    frametitlefont=\normalfont\bfseries, % 标题字体设置
    frametitlealignment=\raggedright % 标题居中
]{redbox}
% 设置页面的环境,a4纸张大小,适当调整边距
\usepackage[a4paper,left=10mm,right=10mm,top=15mm,bottom=15mm]{geometry}
\title{\Huge \textbf{MySQL 必知必会}}  % 主标题加大字体
\author{Red Bamboo}
\date{fall 2024}


\begin{document}
\begin{titlepage}
    \begin{center}
    \vspace*{2cm}  % 调整标题与页面顶端的距离,使其更居中

    % 主标题
    \Huge
    \textbf{MySQL 必知必会}
    
    \vspace{10cm}  % 增大标题和作者部分之间的空白,保持整体平衡感

    % 作者信息
    \Large
    \LaTeX \ by Red Bamboo\\
    Jilin University
    
    \vspace{0.5cm}  % 保持适当的垂直空白
    
    % 日期
    fall 2024
    
    \end{center}
\end{titlepage}

\newpage %另起一页
\tableofcontents
\newpage

\section{数据库基础}
\begin{redbox}[frametitle={Defination 1}]
\begin{enumerate}
	\item \textbf{数据库(database)} 保存有组织的数据的容器(通常是一个文
    件或一组文件)
	\item \textbf{表(table)} 某种特定类型数据的结构化清单。
	\item \textbf{模式(schema)} 关于数据库和表的布局及特性的信息。
	\item \textbf{列(column)} 表中的一个字段。所有表都是由一个或多个列组
    成的。
    \item \textbf{数据类型(datatype)} 所容许的数据的类型。每个表列都有相
    应的数据类型,它限制(或容许)该列中存储的数据。
    \item \textbf{行(row)} 表中的一个记录。
    \item \textbf{主键(primary key)}一列(或一组列),其值能够唯一区分表
    中每个行。
\end{enumerate}
    
\end{redbox}
\section{连接MySQL}

在操作系统命令提示符下输入mysql,可以输入以下指令:
\begin{listing}[htb]
	\caption{连接mysql}
	\label{code:connectmysql}
\begin{minted}{mysql}
        mysql -u hzz -p
\end{minted}
\end{listing}

然后在接下来的对话框中输入密码即可,输入密码完成后会返回如下:

\begin{listing}[htb]
	\caption{连接mysql后的回复}
	\label{code:respondforconnectmysql}
	\begin{minted}{mysql}
        Welcome to the MySQL monitor.  Commands end with ; or \g.
        Your MySQL connection id is 38
        Server version: 8.0.39 MySQL Community Server - GPL
        Type 'help;' or '\h' for help. Type '\c' to clear the current input statement.
        \end{minted}
\end{listing}

\section{使用MySQL}

\subsection{选择数据库}

首先需要创建一个数据库:
\begin{listing}[H]
	\caption{创建数据库}
	\label{code:createdatabase}
	\begin{minted}{mysql}
        create crashcourse;
        \end{minted}
\end{listing}

然后调用这个数据库:

\begin{listing}[H]
	\caption{调用数据库}
	\label{code:usedatabase}
	\begin{minted}{mysql}
        use sakila;
\end{minted}
\end{listing}

若使用成功会返回:

\begin{listing}[H]
	\caption{调用数据库返回值}
	\label{code:respondforusedatabase}
	\begin{minted}{mysql}
        Database changed
\end{minted}
\end{listing}

\subsection{展示数据和表}

展示数据库可以用如下命令:

\begin{listing}[H]
	\caption{显示数据库}
	\label{code:showdatabase}
	\begin{minted}{mysql}
        show databases;
\end{minted}
\end{listing}


此时会返回可用数据库的一个列表:

\begin{listing}[H]
	\caption{数据库列表}
	\label{code:databaselist}
	\begin{minted}{mysql}
        +--------------------+
        | Database           |
        +--------------------+
        | crashcourse        |
        | demo               |
        | information_schema |
        | mysql              |
        | performance_schema |
        | sakila             |
        | sys                |
        | world              |
        +--------------------+
        8 rows in set (0.00 sec)
\end{minted}
\end{listing}

查看一个数据库中的表可以用如下命令:

\begin{listing}[H]
	\caption{显示数据库}
	\label{code:showtable}
	\begin{minted}{mysql}
        show tables;
\end{minted}
\end{listing}

此时会返回可用表的一个列表:

\begin{listing}[H]
	\caption{表列表}
	\label{code:tablelist}
	\begin{minted}{mysql}
        +----------------------------+
        | Tables_in_sakila           |
        +----------------------------+
        | actor                      |
        | actor_info                 |
        | address                    |
        | category                   |
        | city                       |
        | country                    |
        | customer                   |
        | customer_list              |
        | film                       |
        | film_actor                 |
        | film_category              |
        | film_list                  |
        | film_text                  |
        | inventory                  |
        | language                   |
        | nicer_but_slower_film_list |
        | payment                    |
        | rental                     |
        | sales_by_film_category     |
        | sales_by_store             |
        | staff                      |
        | staff_list                 |
        | store                      |
        +----------------------------+
        23 rows in set (0.00 sec)
\end{minted}
\end{listing}

SHOW也可以用来显示表列:

\begin{listing}[H]
	\caption{显示列}
	\label{code:showcolumn}
	\begin{minted}{mysql}
        show columns from actor;
\end{minted}
\end{listing}

此时会返回可用列的一个列表:

\begin{listing}[H]
	\caption{表列}
	\label{code:columnlist}
	\begin{minted}{mysql}
+-------------+-------------------+------+-----+-------------------+-----------------------------------------------+
| Field       | Type              | Null | Key | Default           | Extra                                         |
+-------------+-------------------+------+-----+-------------------+-----------------------------------------------+
| actor_id    | smallint unsigned | NO   | PRI | NULL              | auto_increment                                |
| first_name  | varchar(45)       | NO   |     | NULL              |                                               |
| last_name   | varchar(45)       | NO   | MUL | NULL              |                                               |
| last_update | timestamp         | NO   |     | CURRENT_TIMESTAMP | DEFAULT_GENERATED on update CURRENT_TIMESTAMP |
+-------------+-------------------+------+-----+-------------------+-----------------------------------------------+
4 rows in set (0.00 sec)
\end{minted}
\end{listing}
注意,describe与show columns是等价命令,也就是如下命令也可以返回上面的结果:
\begin{listing}[H]
	\caption{describe}
	\label{code:describe等价命令}
	\begin{minted}{mysql}
        describe actor;
\end{minted}
\end{listing}

show 还有其他语句:
\begin{listing}[H]
	\caption{show的其他语句}
	\label{code:show的其他语句}
	\begin{minted}{mysql}
        show status;                    #用于显示广泛的服务器状态信息;
        show grants;                    #用来显示授予用户(所有用户或特定用户)的安全权限;
        show errors;                    #用来显示服务器错误消息。
        show warnings;                  #用来显示服务器警告消息。
\end{minted}
\end{listing}

\section{检索数据}
\subsection{检索单个列}
首先是先检索表的单列:
\begin{listing}[H]
	\caption{select单列}
	\label{code:selectcolumn}
	\begin{minted}{mysql}
        select name from language;
\end{minted}
\end{listing}

输出的结果为:
\begin{listing}[H]
	\caption{select单列结果}
	\label{code:resultofselectcolumn}
	\begin{minted}{mysql}
        +----------+
        | name     |
        +----------+
        | English  |
        | Italian  |
        | Japanese |
        | Mandarin |
        | French   |
        | German   |
        +----------+
        6 rows in set (0.00 sec)
\end{minted}
\end{listing}
\begin{orangebox}[frametitle={Tips 4.1}]
\begin{enumerate}
    \item 在没有排序的情况下,这个数据顺序没有任何意义。
    \item 多条SQL语句必须以分号(;)分隔。MySQL不需要在单条SQL语句后加分号。(但尽可能加上分号)
    \item SQL语句不区分大小写(可以对所有SQL关键字使用大写,而对所有列和表名使用小写)。
    \item 在处理SQL语句时,其中所有空格都被忽略。
\end{enumerate}
\end{orangebox}
\subsection{检索多个列}
从一个表中检索多个列必须在SELECT关键字后给出多个列名,列名之间必须以逗号分隔:
\begin{listing}[H]
	\caption{select多列}
	\label{code:selectcolumns}
	\begin{minted}{mysql}
        select language_id,name,last_update from language;
\end{minted}
\end{listing}

输出的结果为:

\begin{listing}[H]
	\caption{select多列结果}
	\label{code:resultofselectcolumns}
	\begin{minted}{mysql}
        +-------------+----------+---------------------+
        | language_id | name     | last_update         |
        +-------------+----------+---------------------+
        |           1 | English  | 2006-02-15 05:02:19 |
        |           2 | Italian  | 2006-02-15 05:02:19 |
        |           3 | Japanese | 2006-02-15 05:02:19 |
        |           4 | Mandarin | 2006-02-15 05:02:19 |
        |           5 | French   | 2006-02-15 05:02:19 |
        |           6 | German   | 2006-02-15 05:02:19 |
        +-------------+----------+---------------------+
        6 rows in set (0.00 sec)
\end{minted}
\end{listing}

\begin{orangebox}[frametitle={Tips 4.2}]
\begin{enumerate}
    \item 选择多个列时,一定要在列名之间加上逗号,但最后一个列名后不加。
    \item SQL语句一般返回原始的、无格式的数据。数据的格式化是一个表示问题,而不是一个检索问题。
\end{enumerate}
\end{orangebox}

\subsection{检索所有列}
SELECT语句检索所有的列可以通过星号(*)通配符来达到,如下所示:
\begin{listing}[H]
	\caption{select所有列}
	\label{code:selectallcolumns}
	\begin{minted}{mysql}
        SELECT * FROM store;
\end{minted}
\end{listing}

输出的结果为:

\begin{listing}[H]
	\caption{select所有列结果}
	\label{code:resultofselectallcolumns}
	\begin{minted}{mysql}
        +----------+------------------+------------+---------------------+
        | store_id | manager_staff_id | address_id | last_update         |
        +----------+------------------+------------+---------------------+
        |        1 |                1 |          1 | 2006-02-15 04:57:12 |
        |        2 |                2 |          2 | 2006-02-15 04:57:12 |
        +----------+------------------+------------+---------------------+
        2 rows in set (0.00 sec)
\end{minted}
\end{listing}
\begin{orangebox}[frametitle={Tips 4.3}]
    \begin{enumerate}
        \item 虽然使用通配符可能会使你自己省事,但检索不需要的列通常会降低检索和应用程序的性能。
        \item 使用通配符能检索出名字未知的列。
    \end{enumerate}
    \end{orangebox}
\subsection{检索不同的行(DISTINCT)}

只返回不同的值可以用DISTINCT,如以下所示:

\begin{listing}[H]
	\caption{select不同行}
	\label{code:selectdifferentrows}
	\begin{minted}{mysql}
        SELECT DISTINCT staff_id FROM rental;
\end{minted}
\end{listing}

输出的结果为:

\begin{listing}[H]
	\caption{select所有列结果}
	\label{code:resultofselectdifferent rows}
	\begin{minted}{mysql}
        +----------+
        | staff_id |
        +----------+
        |        1 |
        |        2 |
        +----------+
        2 rows in set (0.00 sec)
\end{minted}
\end{listing}

但如果直接SELECT,不加DISTINCT就会有16044条。

\begin{orangebox}[frametitle={Tips 4.4}]
    \textbf {不能部分使用DISTINCT} ,DISTINCT关键字应用于所有列而不仅是前置它的列。
\end{orangebox}

\subsection{限制结果(LIMIT)}

为了返回第一行或前几行,可使用LIMIT子句:
\begin{listing}[H]
	\caption{LIMIT子句}
	\label{code:limitresult}
	\begin{minted}{mysql}
        SELECT actor_id FROM actor_info LIMIT 5;
    \end{minted}
\end{listing}

输出的结果为:

\begin{listing}[H]
	\caption{LIMIT的结果}
	\label{code:resultlimitresult}
	\begin{minted}{mysql}
        +----------+
        | actor_id |
        +----------+
        |        1 |
        |        2 |
        |        3 |
        |        4 |
        |        5 |
        +----------+
        5 rows in set (1.15 sec)
\end{minted}
\end{listing}
这个结果就是返回开始的5行

如果要返回从第9行开始的5行,可以执行以下命令:
\begin{listing}[H]
	\caption{从特定行开始的LIMIT子句}
	\label{code:distinctlimitresult}
	\begin{minted}{mysql}
        SELECT actor_id FROM actor_info LIMIT 9,5;
    \end{minted}
\end{listing}

输出的结果为:

\begin{listing}[H]
	\caption{从特定行开始的LIMIT的结果}
	\label{code:resultdistinctlimitresult}
	\begin{minted}{mysql}
        +----------+
        | actor_id |
        +----------+
        |       10 |
        |       11 |
        |       12 |
        |       13 |
        |       14 |
        +----------+
        5 rows in set (1.20 sec)
\end{minted}
\end{listing}

\begin{orangebox}[frametitle={Tips 4.5}]
    \begin{enumerate}
        \item 检索出来的第一行为行0而不是行1。因此,LIMIT 1, 1将检索出第二行而不是第一行。
        \item 在行数不够时,MySQL将只返回它能返回的那么多行。
    \end{enumerate}
\end{orangebox}

\subsection{使用完全限定的表名}
可以使用限定列列名来索引列,以下两条命令与第一条本章第一条的结果完全相同:
\begin{listing}[H]
	\caption{限定表名}
	\label{code:distincttablename}
	\begin{minted}{mysql}
        SELECT language.name FROM language;
        SELECT language.name FROM sakila.language;
\end{minted}
\end{listing}

\section{排序数据}
\subsection{普通排序}
之前SQL语句返回数据库表的单列,检索出的数据并不是随机显示的,而是以它在底层表中出现的顺序显示。但是,如果数据后来进行过更新或删除,则此顺
序将会受到MySQL重用回收存储空间的影响。
\begin{redbox}[frametitle={Defination 5.1 子句(clause)}]
    SQL语句由子句构成,有些子句是必需的,而有的是可选的。一个子句通常由一个关键字和所提供的数据组成。
\end{redbox}

因此,为了排序SELECT语句检索出来的数据,我们可以使用ORDER BY语句:
\begin{listing}[H]
	\caption{ORDER BY子句}
	\label{code:orderby}
	\begin{minted}{mysql}
        SELECT city FROM city ORDER BY city_id LIMIT 10;
    \end{minted}
\end{listing}

输出的结果为:

\begin{listing}[H]
	\caption{ORDER BY子句的结果}
	\label{code:orderbyresult}
	\begin{minted}{mysql}
        +----------------------+
        | city                 |
        +----------------------+
        | A Coru?a (La Coru?a) |
        | Abha                 |
        | Abu Dhabi            |
        | Acu?a                |
        | Adana                |
        | Addis Abeba          |
        | Aden                 |
        | Adoni                |
        | Ahmadnagar           |
        | Akishima             |
        +----------------------+
        10 rows in set (0.00 sec)
\end{minted}
\end{listing}

\begin{orangebox}[frametitle={Tips 5.1}]
        通常ORDER BY子句中使用的列是为显示所选择的列。但是可以通过非选择列进行排序。
\end{orangebox}

\subsection{按多列排序}
按多个列排序,只要指定列名,列名之间用逗号分开即可:
\begin{listing}[H]
	\caption{多列排序}
	\label{code:multiplecolumnsorder}
	\begin{minted}{mysql}
        SELECT last_name,first_name,store_id FROM customer ORDER BY last_name,first_name LIMIT 10;
    \end{minted}
\end{listing}

输出的结果为:

\begin{listing}[H]
	\caption{多列排序的结果}
	\label{code:multiplecolumnsorderresult}
	\begin{minted}{mysql}
        +-----------+------------+----------+
        | last_name | first_name | store_id |
        +-----------+------------+----------+
        | ABNEY     | RAFAEL     |        1 |
        | ADAM      | NATHANIEL  |        1 |
        | ADAMS     | KATHLEEN   |        2 |
        | ALEXANDER | DIANA      |        1 |
        | ALLARD    | GORDON     |        1 |
        | ALLEN     | SHIRLEY    |        2 |
        | ALVAREZ   | CHARLENE   |        2 |
        | ANDERSON  | LISA       |        2 |
        | ANDREW    | JOSE       |        1 |
        | ANDREWS   | IDA        |        2 |
        +-----------+------------+----------+
        10 rows in set (0.00 sec)
\end{minted}
\end{listing}

\subsection{指定排序方向}
\begin{orangebox}[frametitle={Tips 5.3}]
    \begin{enumerate}
        \item 默认排序是升序排序(从A到Z)。
        \item 降序排序(从Z到A)必须指定DESC关键字。
        \item 在多个列上降序排序,必须对每个列指定DESC关键字。(与DESC对应的是ASC)
        \item 在字典(dictionary)排序顺序中,A被视为与a相同,这是MySQL(和大多数数据库管理系统)的默认行为。(也就是不区分A和a)。
        \item ORDER BY子句位于FROM子句之后,LIMIT位于ORDER BY之后。
    \end{enumerate}
\end{orangebox}
降序排列的SQL语句如下所示:
\begin{listing}[H]
	\caption{降序排列}
	\label{code:descsorder}
	\begin{minted}{mysql}
        SELECT last_name,first_name,last_update FROM actor ORDER BY last_name DESC LIMIT 10;
    \end{minted}
\end{listing}

输出的结果为:

\begin{listing}[H]
	\caption{降序排列的结果}
	\label{code:descsorderresult}
	\begin{minted}{mysql}
        +-------------+------------+---------------------+
        | last_name   | first_name | last_update         |
        +-------------+------------+---------------------+
        | ZELLWEGER   | JULIA      | 2006-02-15 04:34:33 |
        | ZELLWEGER   | CAMERON    | 2006-02-15 04:34:33 |
        | ZELLWEGER   | MINNIE     | 2006-02-15 04:34:33 |
        | WRAY        | CAMERON    | 2006-02-15 04:34:33 |
        | WOOD        | FAY        | 2006-02-15 04:34:33 |
        | WOOD        | UMA        | 2006-02-15 04:34:33 |
        | WITHERSPOON | ANGELA     | 2006-02-15 04:34:33 |
        | WINSLET     | FAY        | 2006-02-15 04:34:33 |
        | WINSLET     | RIP        | 2006-02-15 04:34:33 |
        | WILSON      | WILL       | 2006-02-15 04:34:33 |
        +-------------+------------+---------------------+
        10 rows in set (0.00 sec)
\end{minted}
\end{listing}

如果多列降序排列,可以参照以下SQL语句:
\begin{listing}[H]
	\caption{多列降序排列}
	\label{code:muldescsorder}
	\begin{minted}{mysql}
        SELECT amount,rental_id,staff_id FROM payment ORDER BY amount DESC,rental_id LIMIT 10;
    \end{minted}
\end{listing}

输出的结果为:

\begin{listing}[H]
	\caption{多列降序排序的结果}
	\label{code:muldescsorderresult}
	\begin{minted}{mysql}
        +--------+-----------+----------+
        | amount | rental_id | staff_id |
        +--------+-----------+----------+
        |  11.99 |       106 |        2 |
        |  11.99 |      2166 |        1 |
        |  11.99 |      3973 |        1 |
        |  11.99 |      4383 |        2 |
        |  11.99 |      8831 |        2 |
        |  11.99 |     11479 |        2 |
        |  11.99 |     14759 |        1 |
        |  11.99 |     14763 |        2 |
        |  11.99 |     15415 |        2 |
        |  11.99 |     16040 |        2 |
        +--------+-----------+----------+
        10 rows in set (0.00 sec)
\end{minted}
\end{listing}

这种方法(ORDER BY和LIMIT的组合)的一个变种就是直接可以检索出极大值或者极小值,由于上面是极大值,下面我以极小值举例子:
\begin{listing}[H]
	\caption{检索极值}
	\label{code:findmaxormin}
	\begin{minted}{mysql}
        SELECT amount FROM payment ORDER BY amount LIMIT 1;
    \end{minted}
\end{listing}

输出的结果为:

\begin{listing}[H]
	\caption{检索极值的结果}
	\label{code:findmaxorminresult}
	\begin{minted}{mysql}
        +--------+
        | amount |
        +--------+
        |   0.00 |
        +--------+
        1 row in set (0.00 sec)
        \end{minted}
\end{listing}

\section{过滤数据}
\subsection{WHERE 子句}
    
\begin{redbox}[frametitle={Defination 6.1 搜索条件(search criteria)}]
        搜索条件也称为过滤条件(filter condition),只检索所需数据需要指定搜索条件(search criteria)。
\end{redbox}
SELECT语句中,数据根据WHERE子句中指定的搜索条件进行过滤,命令如下:
\begin{listing}[H]
	\caption{WHERE子句}
	\label{code:whereclause}
	\begin{minted}{mysql}
         SELECT customer_id,staff_id,amount FROM payment WHERE staff_id = 1 LIMIT 10;
        \end{minted}
\end{listing}

输出的结果为:

\begin{listing}[H]
	\caption{WHERE子句的结果}
	\label{code:whereclauseresult}
	\begin{minted}{mysql}
         +-------------+----------+--------+
         | customer_id | staff_id | amount |
         +-------------+----------+--------+
         |           1 |        1 |   2.99 |
         |           1 |        1 |   0.99 |
         |           1 |        1 |   5.99 |
         |           1 |        1 |   4.99 |
         |           1 |        1 |   4.99 |
         |           1 |        1 |   3.99 |
         |           1 |        1 |   5.99 |
         |           1 |        1 |   4.99 |
         |           1 |        1 |   4.99 |
         |           1 |        1 |   7.99 |
         +-------------+----------+--------+
         10 rows in set (0.00 sec)

        \end{minted}
\end{listing}

\begin{orangebox}[frametitle={Tips 6.1}]
        \begin{enumerate}
            \item 同时使用ORDER BY和WHERE子句时,ORDER BY位于WHERE之后。
            \item WHERE子句在表名(FROM子句)之后。
            \item 单引号用来限定字符串。如果将值与串类型的列进行比较,则需要限定引号。用来与数值列进行比较的值不用引号。
            \item NULL具有特殊含义,在过滤使用不匹配(<>,!=)过滤选择出不具有特定值的行的时候,并不会返回空值,因此需要空值查询。
        \end{enumerate}
\end{orangebox}


\begin{table}[H]
        \caption{\textbf{WHERE子句操作符}}%标题
        \centering%把表居中
        \begin{tabular}{ll}%四个c代表该表一共四列,内容全部居中
        \toprule%第一道横线
        操作符&说明\\
        \midrule%第二道横线 
        =       &       等于                    \\
        <>      &       不等于                  \\
        !=     &       不等于                  \\
        <       &       小于                    \\
        <=      &       小于等于                \\
        >       &       大于                   \\
        >=      &       大于等于               \\
        BETWEEN &       在指定的两个值之间      \\
        \bottomrule%第三道横线
        \end{tabular}
\end{table}  

\subsection{检查单个值}
首先是用等号的WHERE子句:
\begin{listing}[H]
	\caption{用等号的WHERE子句}
	\label{code:equalwhereclause}
	\begin{minted}{mysql}
                SELECT customer_id,email,address_id FROM customer WHERE customer_id = 7;
        \end{minted}
\end{listing}

输出的结果为:

\begin{listing}[H]
	\caption{用等号的WHERE子句的结果}
	\label{code:equalwhereclauseresult}
	\begin{minted}{mysql}
         +-------------+---------------------------------+------------+
         | customer_id | email                           | address_id |
         +-------------+---------------------------------+------------+
         |           7 | MARIA.MILLER@sakilacustomer.org |         11 |
         +-------------+---------------------------------+------------+
         1 row in set (0.00 sec)

        \end{minted}
\end{listing}

接着可以用小于、小于等于、大于、大于等于的WHERE子句进行搜索:
\begin{listing}[H]
	\caption{用大于的WHERE子句}
	\label{code:unequalwhereclause}
	\begin{minted}{mysql}
         SELECT amount,payment_date,last_update FROM payment 
         WHERE amount > 11.00 
         ORDER BY amount DESC 
         LIMIT 5;
        \end{minted}
\end{listing}

输出的结果为:

\begin{listing}[H]
	\caption{用大于的WHERE子句的结果}
	\label{code:unequalwhereclauseresult}
	\begin{minted}{mysql}
         +--------+---------------------+---------------------+
         | amount | payment_date        | last_update         |
         +--------+---------------------+---------------------+
         |  11.99 | 2005-07-29 22:37:41 | 2006-02-15 22:12:31 |
         |  11.99 | 2005-08-21 23:34:00 | 2006-02-15 22:13:08 |
         |  11.99 | 2005-08-23 22:19:33 | 2006-02-15 22:14:00 |
         |  11.99 | 2005-05-25 18:18:19 | 2006-02-15 22:14:00 |
         |  11.99 | 2005-08-22 23:48:56 | 2006-02-15 22:14:07 |
         +--------+---------------------+---------------------+
         5 rows in set (0.00 sec)

        \end{minted}
\end{listing}

\subsection{不匹配检查}
如下是采用不匹配检查的SQL语句,其中<>与!=等价:
\begin{listing}[H]
	\caption{用不匹配检查的WHERE子句}
	\label{code:unmatchwhereclause}
	\begin{minted}{mysql}
         SELECT name,last_update FROM language WHERE name <> 'Japanese';
        \end{minted}
\end{listing}

输出的结果为:

\begin{listing}[H]
	\caption{用不匹配检查的WHERE子句的结果}
	\label{code:unmatchwhereclauseresult}
	\begin{minted}{mysql}
         +----------+---------------------+
         | name     | last_update         |
         +----------+---------------------+
         | English  | 2006-02-15 05:02:19 |
         | Italian  | 2006-02-15 05:02:19 |
         | Mandarin | 2006-02-15 05:02:19 |
         | French   | 2006-02-15 05:02:19 |
         | German   | 2006-02-15 05:02:19 |
         +----------+---------------------+
         5 rows in set (0.00 sec)

        \end{minted}
\end{listing}

\subsection{范围值检查}

范围值检查需要用BETWEEN,如下是使用BETWEEN的SQL语句:
\begin{listing}[H]
	\caption{用BETWEEN的WHERE子句}
	\label{code:betweenwhereclause}
	\begin{minted}{mysql}
         SELECT amount,payment_date,last_update FROM payment 
         WHERE amount BETWEEN 10 AND 11
         ORDER BY amount DESC 
         LIMIT 5;
        \end{minted}
\end{listing}

输出的结果为:

\begin{listing}[H]
	\caption{用BETWEEN的WHERE子句的结果}
	\label{code:betweenwhereclauseresult}
	\begin{minted}{mysql}
         +--------+---------------------+---------------------+
         | amount | payment_date        | last_update         |
         +--------+---------------------+---------------------+
         |  10.99 | 2005-07-30 13:47:43 | 2006-02-15 22:12:30 |
         |  10.99 | 2005-07-27 20:23:12 | 2006-02-15 22:12:30 |
         |  10.99 | 2005-08-01 06:50:26 | 2006-02-15 22:12:31 |
         |  10.99 | 2005-06-21 01:04:35 | 2006-02-15 22:12:33 |
         |  10.99 | 2005-07-09 21:55:19 | 2006-02-15 22:12:35 |
         +--------+---------------------+---------------------+
         5 rows in set (0.00 sec)
        \end{minted}
\end{listing}

\subsection{空值检查}

\begin{redbox}[frametitle={Defination 6.5 NULL 无值(no value)}]
        在一个列不包含值时,称其为包含空值NULL,它与字段包含0、空字符串或仅仅包含空格不同。
\end{redbox}
空值检查的SQL语句如下:
\begin{listing}[H]
	\caption{空值检查的WHERE子句}
	\label{code:isnullnwhereclause}
	\begin{minted}{mysql}
         SELECT rental_date,return_date,staff_id 
         FROM rental 
         WHERE return_date IS NULL 
         LIMIT 5;
        \end{minted}
\end{listing}

输出的结果为:

\begin{listing}[H]
	\caption{空值检查的WHERE子句的结果}
	\label{code:isnullwhereclauseresult}
	\begin{minted}{mysql}
         +---------------------+-------------+----------+
         | rental_date         | return_date | staff_id |
         +---------------------+-------------+----------+
         | 2006-02-14 15:16:03 | NULL        |        1 |
         | 2006-02-14 15:16:03 | NULL        |        1 |
         | 2006-02-14 15:16:03 | NULL        |        1 |
         | 2006-02-14 15:16:03 | NULL        |        2 |
         | 2006-02-14 15:16:03 | NULL        |        1 |
         +---------------------+-------------+----------+
         5 rows in set (0.00 sec)
        \end{minted}
\end{listing}

\section{数据过滤}

\subsection{操作符}
\begin{redbox}[frametitle={Defination 7.1 操作符(operator)}]
        用来联结或改变WHERE子句中的子句的关键字。也称为逻辑操作符(logical operator)。
\end{redbox}

\subsection{AND操作符}
AND操作符给WHERE子句附加条件,参考以下SQL语句:

\begin{listing}[H]
	\caption{含AND操作符WHERE子句}
	\label{code:andwhereclause}
	\begin{minted}{mysql}
         SELECT amount,customer_id,staff_id 
         FROM payment 
         WHERE amount BETWEEN 10 AND 11 
         AND staff_id = 1 
         AND customer_id BETWEEN 0 AND 100;
        \end{minted}
\end{listing}

输出的结果为:

\begin{listing}[H]
	\caption{含AND操作符WHERE子句的结果}
	\label{code:andwhereclauseresult}
	\begin{minted}{mysql}
         +--------+-------------+----------+
         | amount | customer_id | staff_id |
         +--------+-------------+----------+
         |  10.99 |          21 |        1 |
         |  10.99 |          33 |        1 |
         |  10.99 |          45 |        1 |
         |  10.99 |          54 |        1 |
         |  10.99 |          76 |        1 |
         |  10.99 |          78 |        1 |
         |  10.99 |          86 |        1 |
         +--------+-------------+----------+
         7 rows in set (0.01 sec)
        \end{minted}
\end{listing}
\begin{redbox}[frametitle={Defination 7.2 AND操作符}]
        用在WHERE子句中的关键字,用来指示检索满足所有给定条件的行。
\end{redbox}

\subsection{OR操作符}
OR操作符指示MySQL检索匹配任一条件的行,SQL命令如下所示:
\begin{listing}[H]
	\caption{含OR操作符WHERE子句}
	\label{code:orwhereclause}
        \begin{minted}{mysql}
        SELECT payment_date,rental_id,amount 
        FROM payment 
        WHERE amount = 0.99 
        OR rental_id = 76 
        LIMIT 10;
        \end{minted}
\end{listing}

输出的结果为:

\begin{listing}[H]
	\caption{含OR操作符WHERE子句的结果}
	\label{code:orwhereclauseresult}
\begin{minted}{mysql}
        +---------------------+-----------+--------+
        | payment_date        | rental_id | amount |
        +---------------------+-----------+--------+
        | 2005-05-25 11:30:37 |        76 |   2.99 |
        | 2005-05-28 10:35:23 |       573 |   0.99 |
        | 2005-06-15 18:02:53 |      1422 |   0.99 |
        | 2005-06-18 13:33:59 |      2363 |   0.99 |
        | 2005-07-28 17:33:39 |      8074 |   0.99 |
        | 2005-07-28 19:20:07 |      8116 |   0.99 |
        | 2005-08-02 18:01:38 |     11367 |   0.99 |
        | 2005-08-18 03:57:29 |     12250 |   0.99 |
        | 2005-08-19 09:55:16 |     13068 |   0.99 |
        | 2005-08-21 23:33:57 |     14762 |   0.99 |
        +---------------------+-----------+--------+
        10 rows in set (0.00 sec)
\end{minted}
\end{listing}

\subsection{计算次序}
\begin{orangebox}[frametitle={Tips 7.4}]
        \begin{enumerate}
            \item 同时有AND和OR的WHERE子句,优先处理AND,因为AND在计算次序中优先级更高
            \item 如果需要改变优先级,需要圆括号明确地分组操作符。(即使默认次序是正确的情况下)
        \end{enumerate}
\end{orangebox}
例如,我要查询actor\_id为1的并且film\_id为166或277的记录,如果写成以下这样:
\begin{listing}[H]
	\caption{错误次序的WHERE子句}
	\label{code:faultwhereclause}
\begin{minted}{mysql}
        SELECT * FROM film_actor 
        WHERE actor_id =1 AND film_id = 166 
        OR film_id = 277;
\end{minted}
\end{listing}

输出的结果为:

\begin{listing}[H]
	\caption{错误次序的WHERE子句的结果}
	\label{code:faultwhereclauseresult}
\begin{minted}{mysql}
        +----------+---------+---------------------+
        | actor_id | film_id | last_update         |
        +----------+---------+---------------------+
        |        1 |     166 | 2006-02-15 05:05:03 |
        |        1 |     277 | 2006-02-15 05:05:03 |
        |       24 |     277 | 2006-02-15 05:05:03 |
        |       37 |     277 | 2006-02-15 05:05:03 |
        |      107 |     277 | 2006-02-15 05:05:03 |
        |      115 |     277 | 2006-02-15 05:05:03 |
        +----------+---------+---------------------+
        6 rows in set (0.00 sec)
\end{minted}
\end{listing}
此时显然是不对的,我查询成了,actor\_id为1且film\_id为166或者film\_id为277的情况,因此在这种情况下应该将语句改为:
\begin{listing}[H]
	\caption{正确次序的WHERE子句}
	\label{code:rightwhereclause}
\begin{minted}{mysql}
        SELECT * FROM film_actor 
        WHERE actor_id =1 
        AND (film_id = 166 OR film_id = 277);
\end{minted}
\end{listing}

输出的结果为:

\begin{listing}[H]
	\caption{正确次序的WHERE子句的结果}
	\label{code:rightwhereclauseresult}
\begin{minted}{mysql}
        +----------+---------+---------------------+
        | actor_id | film_id | last_update         |
        +----------+---------+---------------------+
        |        1 |     166 | 2006-02-15 05:05:03 |
        |        1 |     277 | 2006-02-15 05:05:03 |
        +----------+---------+---------------------+
        2 rows in set (0.00 sec)
\end{minted}
\end{listing}

\subsection{IN操作符}
IN操作符用来指定条件范围,范围中的每个条件都可以进行匹配,因此IN操作符在一定程度上可以代替OR操作符,我们仍然用上面那个例子:
\begin{listing}[H]
	\caption{含OR操作符WHERE子句}
	\label{code:secondorwhereclause}
\begin{minted}{mysql}
        SELECT * FROM film_actor 
        WHERE actor_id =1 
        AND film_id IN (166,277);
\end{minted}
\end{listing}

输出的结果为:

\begin{listing}[H]
	\caption{含OR操作符WHERE子句的结果}
	\label{code:secondorwhereclauseresult}
\begin{minted}{mysql}
        +----------+---------+---------------------+
        | actor_id | film_id | last_update         |
        +----------+---------+---------------------+
        |        1 |     166 | 2006-02-15 05:05:03 |
        |        1 |     277 | 2006-02-15 05:05:03 |
        +----------+---------+---------------------+
        2 rows in set (0.00 sec)
\end{minted}
\end{listing}
由此可以给出IN操作符的定义:
\begin{redbox}[frametitle={Defination 7.5 IN操作符}]
        WHERE子句中用来指定要匹配值的清单的关键字,功能与OR相当。
\end{redbox}

\begin{orangebox}[frametitle={Tips 7.5}]
        \begin{enumerate}
            \item IN操作符的语法更清楚且更直观。
            \item IN操作符计算的次序更容易管理。
            \item IN操作符一般比OR操作符清单执行更快。
            \item IN操作符可以包含其他SELECT语句,使得能够更动态地建立WHERE子句。
        \end{enumerate}
\end{orangebox}

\subsection{NOT操作符}
\begin{redbox}[frametitle={Defination 7.6 NOT操作符}]
        WHERE子句中用来否定后跟条件的关键字。
\end{redbox}

这个例子改成不是actor\_id为1且film\_id为166或者film\_id为277,并且限制10条,则有:
\begin{listing}[H]
	\caption{含NOT操作符WHERE子句}
	\label{code:notwhereclause}
\begin{minted}{mysql}
        SELECT * FROM film_actor 
        WHERE actor_id =1 
        AND film_id NOT IN (166,277)
        LIMIT 10;
\end{minted}
\end{listing}

输出的结果为:

\begin{listing}[H]
	\caption{含NOT操作符WHERE子句的结果}
	\label{code:notwhereclauseresult}
\begin{minted}{mysql}
        +----------+---------+---------------------+
        | actor_id | film_id | last_update         |
        +----------+---------+---------------------+
        |        1 |       1 | 2006-02-15 05:05:03 |
        |        1 |      23 | 2006-02-15 05:05:03 |
        |        1 |      25 | 2006-02-15 05:05:03 |
        |        1 |     106 | 2006-02-15 05:05:03 |
        |        1 |     140 | 2006-02-15 05:05:03 |
        |        1 |     361 | 2006-02-15 05:05:03 |
        |        1 |     438 | 2006-02-15 05:05:03 |
        |        1 |     499 | 2006-02-15 05:05:03 |
        |        1 |     506 | 2006-02-15 05:05:03 |
        |        1 |     509 | 2006-02-15 05:05:03 |
        +----------+---------+---------------------+
        10 rows in set (0.00 sec)
\end{minted}
\end{listing}

\begin{orangebox}[frametitle={Tips 7.6}]
        MySQL 支 持 使 用 NOT 对 IN 、 BETWEEN 和EXISTS子句取反。
\end{orangebox}


\section{通配符}
\subsection{LIKE操作符}
\begin{redbox}[frametitle={Defination 8.1 通配符(wildcard)}]
        用来匹配值的一部分的特殊字符。
\end{redbox}
\begin{redbox}[frametitle={Defination 8.1 搜索模式(search pattern)}]
        由字面值、通配符或两者组合构成的搜索条件。
\end{redbox}

在搜索子句中使用通配符必须使用LIKE操作符。LIKE指示MySQL,后跟的搜索模式利用通配符匹配而不是直接相等匹配进行比较。

\begin{orangebox}[frametitle={Tips 8.1}]
\begin{enumerate}
        \item LIKE是谓词而不是操作符。
        \item 操作符作为谓词(predicate)时不是操作符。
\end{enumerate}  
\end{orangebox}

\subsection{百分号(\%)通配符}

\begin{redbox}[frametitle={Defination 8.2 百分号(\%)通配符}]
        \%表示任何字符出现任意次数。
\end{redbox}

例如,我要找到address表中district字段所有开头Ha的记录,就可以采取以下SQL语句:
\begin{listing}[H]
	\caption{百分号通配符}
	\label{code:percentsignwildcard}
\begin{minted}{mysql}
        SELECT DISTINCT district FROM address WHERE district LIKE 'Ha%';
\end{minted}
\end{listing}

输出的结果为:

\begin{listing}[H]
	\caption{含百分号通配符的结果}
	\label{code:percentsignwildcardresult}
\begin{minted}{mysql}
        +----------+
        | district |
        +----------+
        | Hamilton |
        | Haryana  |
        | Ha Darom |
        | Haiphong |
        | Haskovo  |
        | Hawalli  |
        | Hanoi    |
        | Hainan   |
        +----------+
        8 rows in set (0.00 sec)
\end{minted}
\end{listing}

\begin{orangebox}[frametitle={Tips 8.2}]
        \begin{enumerate}
                \item \textbf{区分大小写 }  根据MySQL的配置方式,搜索可以是区分大小写的。
                \item \textbf{注意尾空格 } 尾空格可能会干扰通配符匹配。
                \item \textbf{注意NULL } 通配符不可以匹配NULL。
                \item \textbf{匹配字符数 } \%代表搜索模式中给定位置的0个、1个或多个字符。
        \end{enumerate}      
\end{orangebox}

\subsection{下划线(\_)通配符}
\begin{redbox}[frametitle={Defination 8.3 下划线(\_)通配符}]
        下划线\_的用途与\%一样,但下划线只匹配单个字符而不是多个字符。
\end{redbox}

现在我要找到customer中字段名为first\_name中包含开头为ERI的数据,则SQL语句可以是:
\begin{listing}[H]
	\caption{下划线通配符}
	\label{code:underlinewildcard}
\begin{minted}{mysql}
        SELECT first_name,last_name FROM customer WHERE first_name LIKE 'ERI_';
\end{minted}
\end{listing}

输出的结果为:

\begin{listing}[H]
	\caption{含下划线通配符的结果}
	\label{code:underlinewildcardresult}
\begin{minted}{mysql}
        +------------+-----------+
        | first_name | last_name |
        +------------+-----------+
        | ERIN       | DUNN      |
        | ERIC       | ROBERT    |
        | ERIK       | GUILLEN   |
        +------------+-----------+
        3 rows in set (0.00 sec)
\end{minted}
\end{listing}

\section{正则表达式}
\subsection{基本字符匹配}

REGEXP跟的东西为正则表达式,我们查询first\_name字段包含ERI的所有记录,SQL语句可以为:

\begin{listing}[H]
	\caption{基本字符匹配}
	\label{code:matchword}
\begin{minted}{mysql}
        SELECT first_name,last_name FROM customer WHERE first_name REGEXP 'ERI';
\end{minted}
\end{listing}

输出的结果为:

\begin{listing}[H]
	\caption{基本字符匹配的结果}
	\label{code:matchwordresult}
\begin{minted}{mysql}
        +------------+-----------+
        | first_name | last_name |
        +------------+-----------+
        | CATHERINE  | CAMPBELL  |
        | KATHERINE  | RIVERA    |
        | VALERIE    | BLACK     |
        | ERIN       | DUNN      |
        | ERICA      | MATTHEWS  |
        | ERIKA      | PENA      |
        | ERIC       | ROBERT    |
        | FREDERICK  | ISBELL    |
        | ERIK       | GUILLEN   |
        +------------+-----------+
        9 rows in set (0.00 sec)
\end{minted}
\end{listing}

正则表达式中\.表示为任意一个字符,这点可以实现与LIKE相同的效果,可以查找customer表中id为177,277……等记录:

\begin{listing}[H]
	\caption{匹配任意一个字符的正则表达式字符}
	\label{code:matchonewordzhengze}
\begin{minted}{mysql}
        SELECT customer_id,first_name,last_name FROM customer WHERE customer_id REGEXP '.77';
\end{minted}
\end{listing}

输出的结果为:

\begin{listing}[H]
	\caption{含匹配任意一个字符的正则表达式字符的结果}
	\label{code:matchonewordzhengzeresult}
\begin{minted}{mysql}
        +-------------+------------+-----------+
        | customer_id | first_name | last_name |
        +-------------+------------+-----------+
        |         177 | SAMANTHA   | DUNCAN    |
        |         277 | OLGA       | JIMENEZ   |
        |         377 | HOWARD     | FORTNER   |
        |         477 | DAN        | PAINE     |
        |         577 | CLIFTON    | MALCOLM   |
        +-------------+------------+-----------+
\end{minted}
\end{listing}

\begin{orangebox}[frametitle={Tips 9.1.1 LIKE与RECEGP的区别}]
\begin{enumerate}
        \item LIKE匹配整个列。LIKE不会在列值找到被匹配的文本相应的行也不被返回。
        \item REGEXP在列值内进行匹配,REGEXP会找到在列值中被匹配的文本。
\end{enumerate}
\end{orangebox}

\begin{orangebox}[frametitle={Tips 9.1.2 匹配不区分大小写 }]
        正则表达式匹配不区分大小写,为区分大小写,可使用BINARY关键字。
\end{orangebox}

\subsection{OR匹配}

OR条件使用|,比如我在上面的例子搜索ID为177、277、377、477的记录,则有:

\begin{listing}[H]
	\caption{含OR匹配的正则表达式}
	\label{code:ormatchzhengze}
\begin{minted}{mysql}
        SELECT customer_id,first_name,last_name 
        FROM customer 
        WHERE customer_id REGEXP '177|277|377|477';
\end{minted}
\end{listing}

输出的结果为:

\begin{listing}[H]
	\caption{含OR匹配的正则表达式结果}
	\label{code:ormatchzhengzeresult}
\begin{minted}{mysql}
        +-------------+------------+-----------+
        | customer_id | first_name | last_name |
        +-------------+------------+-----------+
        |         177 | SAMANTHA   | DUNCAN    |
        |         277 | OLGA       | JIMENEZ   |
        |         377 | HOWARD     | FORTNER   |
        |         477 | DAN        | PAINE     |
        +-------------+------------+-----------+
        4 rows in set (0.00 sec)
\end{minted}
\end{listing}

\subsection{匹配几个字符之一}
格式是通过指定一组用[和]括起来的字符来完成,可以与上一个输出具有相同的效果:

\begin{listing}[H]
	\caption{匹配几个字符的正则表达式}
	\label{code:mulmatchzhengze}
\begin{minted}{mysql}
        SELECT customer_id,first_name,last_name 
        FROM customer 
        WHERE customer_id REGEXP '[1234]77';
\end{minted}
\end{listing}

输出的结果为:

\begin{listing}[H]
	\caption{匹配几个字符的正则表达式结果}
	\label{code:mulmatchzhengzeresult}
\begin{minted}{mysql}
        +-------------+------------+-----------+
        | customer_id | first_name | last_name |
        +-------------+------------+-----------+
        |         177 | SAMANTHA   | DUNCAN    |
        |         277 | OLGA       | JIMENEZ   |
        |         377 | HOWARD     | FORTNER   |
        |         477 | DAN        | PAINE     |
        +-------------+------------+-----------+
        4 rows in set (0.00 sec)
\end{minted}
\end{listing}

\begin{orangebox}[frametitle={Tips 9.3.1 |与[ ]的对比}]
        [ ]是另一种形式的OR语句。事实上,正则表达式[123]Ton为[1|2|3]Ton的缩写,也可以使用后者。
\end{orangebox}


\begin{orangebox}[frametitle={Tips 9.3.2 否定字符集}]
        在集合的开始处放置一个\^{}即可([\^{}123]却匹配除这些字符外的任何东西)。
\end{orangebox}

\subsection{匹配范围}

\begin{orangebox}[frametitle={Tips 9.4.1 匹配范围}]
\begin{enumerate}
        \item \textbf{数字集合}:[0-9]
        \item \textbf{字母集合}:[a-z]
\end{enumerate}
\end{orangebox}
由此上面的例子可以写成:

\begin{listing}[H]
	\caption{匹配范围的正则表达式}
	\label{code:rangematchzhengze}
\begin{minted}{mysql}
        SELECT customer_id,first_name,last_name 
        FROM customer 
        WHERE customer_id REGEXP '[1-5]77';
\end{minted}
\end{listing}

输出的结果为:

\begin{listing}[H]
	\caption{匹配范围的正则表达式结果}
	\label{code:rangematchzhengzeresult}
\begin{minted}{mysql}
        +-------------+------------+-----------+
        | customer_id | first_name | last_name |
        +-------------+------------+-----------+
        |         177 | SAMANTHA   | DUNCAN    |
        |         277 | OLGA       | JIMENEZ   |
        |         377 | HOWARD     | FORTNER   |
        |         477 | DAN        | PAINE     |
        |         577 | CLIFTON    | MALCOLM   |
        +-------------+------------+-----------+
        5 rows in set (0.00 sec)
\end{minted}
\end{listing}

\subsection{匹配特殊字符}

\begin{redbox}[frametitle={Defination 9.5.1 转义(escaping)}]
        匹配特殊字符,必须用\textbackslash \textbackslash 为前导来查找特殊字符,这种处理就是所谓的转义。
\end{redbox}

\begin{table}[H]
        \caption{\textbf{空白元字符}}%标题
        \centering%把表居中
        \begin{tabular}{cl}
        \toprule%第一道横线
        元字符&说明\\
        \midrule%第二道横线 
        \textbackslash \textbackslash f       &       换页                  \\
        \textbackslash \textbackslash n      &       换行                  \\
        \textbackslash \textbackslash r     &       回车                  \\
        \textbackslash \textbackslash t       &      制表                   \\
        \textbackslash \textbackslash v       &      纵向制表                   \\
        \bottomrule%第三道横线
        \end{tabular}
\end{table}  

我们可以检索包含.CROUSE的邮箱:

\begin{listing}[H]
	\caption{匹配特殊字符的正则表达式}
	\label{code:specialmatchzhengze}
\begin{minted}{mysql}
        SELECT customer_id,first_name,last_name,email 
        FROM customer 
        WHERE email REGEXP '\\.CAS';
\end{minted}
\end{listing}

输出的结果为:

\begin{listing}[H]
	\caption{匹配特殊字符的正则表达式结果}
	\label{code:specialmatchzhengzeresult}
\begin{minted}{mysql}
        +-------------+------------+-----------+--------------------------------------+
        | customer_id | first_name | last_name | email                                |
        +-------------+------------+-----------+--------------------------------------+
        |         207 | GERTRUDE   | CASTILLO  | GERTRUDE.CASTILLO@sakilacustomer.org |
        |         282 | JENNY      | CASTRO    | JENNY.CASTRO@sakilacustomer.org      |
        |         423 | ALFRED     | CASILLAS  | ALFRED.CASILLAS@sakilacustomer.org   |
        +-------------+------------+-----------+--------------------------------------+
        3 rows in set (0.00 sec)
\end{minted}
\end{listing}

\begin{orangebox}[frametitle={Tips 9.5.1 匹配\textbackslash}]
        为了匹配反斜杠(\textbackslash)字符本身,需要使用\textbackslash \textbackslash \textbackslash。
\end{orangebox}

\subsection{匹配字符类}

\begin{redbox}[frametitle={Defination 9.6.1 字符类(character class)}]
        为更方便工作预定义的字符集。
\end{redbox}

\begin{table}[H]
        \caption{\textbf{字符类}}%标题
        \centering%把表居中
        \begin{tabular}{cl}%两列,左边居中,右边左对齐
        \toprule%第一道横线
        类 & 说明 \\
        \midrule%第二道横线 
        \texttt{[:alnum:]} & 任意字母和数字(同[a-zA-Z0-9]) \\
        \texttt{[:alpha:]} & 任意字符(同[a-zA-Z]) \\
        \texttt{[:blank:]} & 任意字符(同[a-zA-Z]) \\
        \texttt{[:cntrl:]} & ASCII控制字符(ASCII 0到31和127) \\
        \texttt{[:digit:]} & 任意数字(同[0-9]) \\
        \texttt{[:graph:]} & 与\texttt{[:print:]}相同,但不包括空格 \\
        \texttt{[:lower:]} & 任意小写字母(同[a-z]) \\
        \texttt{[:print:]} & 任意可打印字符 \\
        \texttt{[:punct:]} & 既不在\texttt{[:alnum:]}又不在\texttt{[:cntrl:]}中的任意字符 \\
        \texttt{[:space:]} & 包括空格在内的任意空白字符(同[\textbackslash\textbackslash f \textbackslash\textbackslash n \textbackslash\textbackslash r \textbackslash\textbackslash t \textbackslash\textbackslash v]) \\
        \texttt{[:upper:]} & 任意大写字母(同[A-Z]) \\
        \texttt{[:xdigit:]} & 任意十六进制数字(同[a-fA-F0-9]) \\
        \bottomrule%第三道横线
        \end{tabular}
\end{table}
 
\subsection{匹配多个实例}

\begin{table}[H]
        \caption{\textbf{重复元字符}}%标题
        \centering%把表居中
        \begin{tabular}{cl}%定义两列,内容居中,并且添加竖直的分隔线
        \toprule%第一道横线
        元字符 & 说明 \\
        \midrule%第二道横线 
        * & 0个或多个匹配 \\
        + & 1个或多个匹配(等于{1,}) \\
        ? & 0个或1个匹配(等于{0,1}) \\
        \{n\} & 指定数目的匹配 \\
        \{n,\} & 不少于指定数目的匹配 \\
        \{n,m\} & 匹配数目的范围(m不超过255) \\
        \bottomrule%第三道横线
        \end{tabular}
\end{table}
例如,可以寻找rating为'PG','PG-13'的记录:

\begin{listing}[H]
	\caption{匹配多个实例的正则表达式}
	\label{code:multimatchzhengze}
\begin{minted}{mysql}
        SELECT FID,title,rating 
        FROM film_list  
        WHERE rating REGEXP 'PG*'  
        LIMIT 10;
\end{minted}
\end{listing}

输出的结果为:

\begin{listing}[H]
	\caption{匹配多个实例的正则表达式结果}
	\label{code:multimatchzhengzeresult}
\begin{minted}{mysql}
        +------+------------------+--------+
        | FID  | title            | rating |
        +------+------------------+--------+
        |    1 | ACADEMY DINOSAUR | PG     |
        |    6 | AGENT TRUMAN     | PG     |
        |    7 | AIRPLANE SIERRA  | PG-13  |
        |    9 | ALABAMA DEVIL    | PG-13  |
        |   12 | ALASKA PHANTOM   | PG     |
        |   13 | ALI FOREVER      | PG     |
        |   18 | ALTER VICTORY    | PG-13  |
        |   19 | AMADEUS HOLY     | PG     |
        |   28 | ANTHEM LUKE      | PG-13  |
        |   33 | APOLLO TEEN      | PG-13  |
        +------+------------------+--------+
\end{minted}
\end{listing}

\subsection{定位符}

\begin{table}[H]
        \caption{\textbf{定位元字符}}%标题
        \centering%把表居中
        \begin{tabular}{cl}%定义两列,内容居中,并且添加竖直的分隔线
        \toprule%第一道横线
        元字符 & 说明 \\
        \midrule%第二道横线 
        \^{} & 文本的开始 \\
        \$ & 文本的结尾 \\
        \texttt{[[:<:]]} & 词的开始 \\
        \texttt{[[:>:]]} & 词的结尾 \\
        \bottomrule%第三道横线
        \end{tabular}
\end{table}

我们尝试完全实现LIKE的功能:

\begin{listing}[H]
	\caption{匹配定位符的正则表达式}
	\label{code:locationmatchzhengze}
\begin{minted}{mysql}
        SELECT first_name, last_name 
        FROM customer 
        WHERE first_name REGEXP '^ERI.$';        
\end{minted}
\end{listing}

\begin{listing}[H]
	\caption{匹配定位符的正则表达式结果}
	\label{code:locationmatchzhengzeresult}
\begin{minted}{mysql}
        +------------+-----------+
        | first_name | last_name |
        +------------+-----------+
        | ERIN       | DUNN      |
        | ERIC       | ROBERT    |
        | ERIK       | GUILLEN   |
        +------------+-----------+
        3 rows in set (0.00 sec)
\end{minted}
\end{listing}


\section{创建计算字段}
\subsection{计算字段}

\begin{redbox}[frametitle={Defination 10.1 字段(field) }]
        基本上与列(column)的意思相同,数据库列一般称为列,字段通常用在计算字段的连接上。
\end{redbox}


\begin{redbox}[frametitle={Defination 10.1 计算字段}]
        计算字段不存在于数据库表中。计算字段是运行时在SELECT语句内创建的。
\end{redbox}

\subsection{拼接字段}
\begin{redbox}[frametitle={Defination 10.2 拼接(concatenate)}]
        将值联结到一起构成单个值,MySQL的SELECT语句中用Concat()函数来拼接两个列。
\end{redbox}

可以采用第八章最后customer表的例子,我们将first\_name字段和laste\_name字段连接起来:
\begin{listing}[H]
	\caption{拼接字段语句}
	\label{code:concatfield}
\begin{minted}{mysql}
        SELECT Concat(first_name,' ',last_name,'(',email,')')  
        FROM customer 
        WHERE first_name LIKE 'ERI_';
\end{minted}
\end{listing}

输出的结果为:

\begin{listing}[H]
	\caption{拼接字段语句的结果}
	\label{code:concatfieldresult}
\begin{minted}{mysql}
        +------------------------------------------------+
        | Concat(first_name,' ',last_name,'(',email,')') |
        +------------------------------------------------+
        | ERIN DUNN(ERIN.DUNN@sakilacustomer.org)        |
        | ERIC ROBERT(ERIC.ROBERT@sakilacustomer.org)    |
        | ERIK GUILLEN(ERIK.GUILLEN@sakilacustomer.org)  |
        +------------------------------------------------+
        3 rows in set (0.00 sec)
\end{minted}
\end{listing}
在此基础上,可以运用RTrim()函数删除多余空格进而整理数据:
\begin{listing}[H]
	\caption{整理拼接字段语句}
	\label{code:treatconcatfield}
\begin{minted}{mysql}
        SELECT Concat(RTrim(first_name),' ',RTrim(last_name),'(',RTrim(email),')')  
        FROM customer 
        WHERE first_name LIKE 'ERI_'
        ORDER BY first_name DESC;
\end{minted}
\end{listing}

输出的结果为:

\begin{listing}[H]
	\caption{整理拼接字段语句的结果}
	\label{code:treatconcatfieldresult}
\begin{minted}{mysql}
        +---------------------------------------------------------------------+
        | Concat(RTrim(first_name),' ',RTrim(last_name),'(',RTrim(email),')') |
        +---------------------------------------------------------------------+
        | ERIN DUNN(ERIN.DUNN@sakilacustomer.org)                             |
        | ERIK GUILLEN(ERIK.GUILLEN@sakilacustomer.org)                       |
        | ERIC ROBERT(ERIC.ROBERT@sakilacustomer.org)                         |
        +---------------------------------------------------------------------+
        3 rows in set (0.00 sec)
\end{minted}
\end{listing}

\begin{orangebox}[frametitle={Tips 10.2}]
        \textbf{Trim函数} MySQL除了支持RTrim()(正如刚才所见,它去掉串右边的空格),还支持LTrim()(去掉串左边的空格)以及Trim()(去掉串左右两边的空格)。   
\end{orangebox}

SELECT的拼接语句生成的字段没有新列的名字,因此我们可以通过别名来实现。
\begin{redbox}[frametitle={Defination 10.2 别名(alias)}]
        一个字段或值的替换名。别名用AS关键字赋予。(别名有时也称为导出列(derived column))
\end{redbox}

那么可以把新生成的字段命名为contact,则有:
\begin{listing}[H]
	\caption{命名拼接字段语句}
	\label{code:renameconcatfield}
\begin{minted}{mysql}
        SELECT Concat(RTrim(first_name),' ',RTrim(last_name),'(',RTrim(email),')')  AS contact
        FROM customer 
        WHERE first_name LIKE 'ERI_'
        ORDER BY first_name DESC;
\end{minted}
\end{listing}

输出的结果为:

\begin{listing}[H]
	\caption{命名拼接字段语句的结果}
	\label{code:renameconcatfieldresult}
\begin{minted}{mysql}
        +-----------------------------------------------+
        | contact                                       |
        +-----------------------------------------------+
        | ERIN DUNN(ERIN.DUNN@sakilacustomer.org)       |
        | ERIK GUILLEN(ERIK.GUILLEN@sakilacustomer.org) |
        | ERIC ROBERT(ERIC.ROBERT@sakilacustomer.org)   |
        +-----------------------------------------------+
        3 rows in set (0.00 sec)
\end{minted}
\end{listing}

\subsection{执行算术计算}
我们可以检索id为7的公司年限和薪水的乘积:
\begin{listing}[H]
	\caption{执行算术计算语句}
	\label{code:conductcalculationclause}
\begin{minted}{mysql}
        SELECT film_id,
               rental_duration,
               replacement_cost,
               rental_duration*replacement_cost AS payment 
        FROM film 
        WHERE film_id=7;
\end{minted}
\end{listing}

输出的结果为:

\begin{listing}[H]
	\caption{执行算术计算语句的结果}
	\label{code:conductcalculationclauseresult}
\begin{minted}{mysql}
        +---------+-----------------+------------------+---------+
        | film_id | rental_duration | replacement_cost | payment |
        +---------+-----------------+------------------+---------+
        |       7 |               6 |            28.99 |  173.94 |
        +---------+-----------------+------------------+---------+
        1 row in set (0.00 sec)
\end{minted}
\end{listing}

\begin{table}[H]
        \caption{\textbf{MySQL算术操作符}}%标题
        \centering%把表居中
        \begin{tabular}{ll}%四个c代表该表一共四列,内容全部居中
        \toprule%第一道横线
        操作符&说明\\
        \midrule%第二道横线 
        +       &       加                    \\
        -      &       减                  \\
        *     &       乘                  \\
        /       &      除                   \\
        \bottomrule%第三道横线
        \end{tabular}
\end{table}  

\begin{orangebox}[frametitle={Tips 10.3.1 测试计算}]
\begin{enumerate}
        \item SELECT可以省略FROM子句以便简单地访问和处理表达式(SELECT3*2;将返回6)。
        \item SELECTNow()利用Now()函数返回当前日期和时间。
\end{enumerate}       
\end{orangebox}

\section{数据处理函数}

\subsection{文本处理函数}

可以运用文本处理函数返回字符串前3个字符:

\begin{listing}[H]
	\caption{执行文本处理函数语句}
	\label{code:conducttextfunctionclause}
\begin{minted}{mysql}
        SELECT film_id, title, LEFT(title, 3) AS sutitle 
        FROM film_text 
        WHERE film_id REGEXP '^[1-5]7$';        
\end{minted}
\end{listing}

输出的结果为:

\begin{listing}[H]
	\caption{执行文本处理函数语句的结果}
	\label{code:conducttextfunctionclauseresult}
\begin{minted}{mysql}
        +---------+-----------------+---------+
        | film_id | title           | sutitle |
        +---------+-----------------+---------+
        |      17 | ALONE TRIP      | ALO     |
        |      27 | ANONYMOUS HUMAN | ANO     |
        |      37 | ARIZONA BANG    | ARI     |
        |      47 | BABY HALL       | BAB     |
        |      57 | BASIC EASY      | BAS     |
        +---------+-----------------+---------+
        5 rows in set (0.00 sec)
\end{minted}
\end{listing}

\begin{table}[H]
        \caption{\textbf{常用的文本处理函数}}%标题
        \centering%把表居中
        \begin{tabular}{ll}%四个c代表该表一共四列,内容全部居中
        \toprule%第一道横线
        函数&说明\\
        \midrule%第二道横线 
        Left()&返回串左边的字符\\
        Length()&返回串的长度\\
        Locate()&找出串的一个子串\\
        Lower()&将串转换为小写\\
        LTrim()&去掉串左边的空格\\
        Right()&返回串右边的字符\\
        RTrim()&去掉串右边的空格\\
        Soundex()&返回串的SOUNDEX值\\
        SubString()&返回子串的字符\\
        Upper()&将串转换为大写\\
        \bottomrule%第三道横线
        \end{tabular}
\end{table}  

\begin{orangebox}[frametitle={Tips 11.1.1 SOUNDEX}]
        SOUNDEX是一个将任何文本串转换为描述其语音表示的字母数字模式的算法。SOUNDEX考虑了类似的发音字符和音节,使得能对串进行发音比较而不是字母比较。    
\end{orangebox}

\subsection{日期和时间处理函数}

DataBase中,日期和时间采用相应的数据类型和特殊的格式存储。

\begin{orangebox}[frametitle={Tips 11.2.1 日期格式}]
        \begin{enumerate}
                \item 日期必须为格式yyyy-mm-dd。
                \item 应该总是使用4位数字的年份,尽管MYSQL支持2位数字的年份,(MySQL处理00-69为2000-2069,处理70-99为1970-1999)。
                \item 如果你想要的仅是日期,则使用Date()。
        \end{enumerate}       
\end{orangebox}

\begin{table}[H]
        \caption{\textbf{常用日期和时间处理函数}}%标题
        \centering%把表居中
        \begin{tabular}{ll}%四个c代表该表一共四列,内容全部居中
        \toprule%第一道横线
        函数&说明\\
        \midrule%第二道横线 
        AddDate()    & 增加一个日期(天、周等) \\ 
        AddTime()    & 增加一个时间(时、分等) \\ 
        CurDate()    & 返回当前日期 \\ 
        CurTime()    & 返回当前时间 \\ 
        Date()       & 返回日期时间的日期部分 \\ 
        DateDiff()   & 计算两个日期之差 \\ 
        Date\_Add()  & 高度灵活的日期运算函数 \\ 
        Date\_Format() & 返回一个格式化的日期或时间串 \\ 
        Day()        & 返回一个日期的天数部分 \\ 
        DayOfWeek()  & 对于一个日期,返回对应的星期几 \\ 
        Hour()       & 返回一个时间的小时部分 \\ 
        Minute()     & 返回一个时间的分钟部分 \\ 
        Month()      & 返回一个日期的月份部分 \\ 
        Now()        & 返回当前日期和时间 \\ 
        Second()     & 返回一个时间的秒部分 \\ 
        Time()       & 返回一个日期时间的时间部分 \\ 
        Year()       & 返回一个日期的年份部分 \\ 
        \bottomrule%第三道横线
        \end{tabular}
\end{table}  

为了说明,现在可以检索address表中30分0秒至30分3秒生成的记录:

\begin{listing}[H]
	\caption{执行日期和时间处理函数语句}
	\label{code:conductdatetimefunctionclause}
\begin{minted}{mysql}
        SELECT address,last_update 
        FROM address 
        WHERE Minute(last_update)=30 
        AND Second(last_update) BETWEEN 0 AND 3;       
\end{minted}
\end{listing}

输出的结果为:

\begin{listing}[H]
	\caption{执行日期和时间处理函数语句的结果}
	\label{code:conductdatetimefunctionclauseresult}
\begin{minted}{mysql}
        +-----------------------------+---------------------+
        | address                     | last_update         |
        +-----------------------------+---------------------+
        | 757 Rustenburg Avenue       | 2014-09-25 22:30:01 |
        | 1892 Nabereznyje Telny Lane | 2014-09-25 22:30:02 |
        | 1368 Maracabo Boulevard     | 2014-09-25 22:30:03 |
        | 368 Hunuco Boulevard        | 2014-09-25 22:30:03 |
        | 486 Ondo Parkway            | 2014-09-25 22:30:02 |
        +-----------------------------+---------------------+
        5 rows in set (0.00 sec)
\end{minted}
\end{listing}

\subsection{数值处理函数}

\begin{table}[H]
        \caption{\textbf{常用数值处理函数}}%标题
        \centering%把表居中
        \begin{tabular}{ll}%四个c代表该表一共四列,内容全部居中
        \toprule%第一道横线
        函数&说明\\
        \midrule%第二道横线 
        Abs()  & 返回一个数的绝对值     \\ 
        Cos()  & 返回一个角度的余弦     \\ 
        Exp()  & 返回一个数的指数值     \\ 
        Mod()  & 返回除操作的余数       \\ 
        Pi()   & 返回圆周率             \\ 
        Rand() & 返回一个随机数         \\ 
        Sin()  & 返回一个角度的正弦     \\ 
        Sqrt() & 返回一个数的平方根     \\ 
        Tan()  & 返回一个角度的正切     \\ 
        \bottomrule%第三道横线
        \end{tabular}
\end{table}  

\section{汇总数据}
\subsection{聚集函数}

\begin{redbox}[frametitle={Defination 12.1 聚集函数(aggregate function) }]
        运行在行组上,计算和返回单个值的函数。
\end{redbox}

\begin{table}[H]
        \caption{\textbf{SQL聚集函数}}%标题
        \centering%把表居中
        \begin{tabular}{ll}%四个c代表该表一共四列,内容全部居中
        \toprule%第一道横线
        函数&说明\\
        \midrule%第二道横线 
        AVG()  & 返回某列的平均值     \\ 
        COUNT()  & 返回某列的行数     \\ 
        MAX() & 返回某列的最大值    \\ 
        MIN()  & 返回某列的最小值       \\ 
        SUM()  & 返回某列值之和             \\ 
        \bottomrule%第三道横线
        \end{tabular}
\end{table}  

\subsection{AVG()函数}

看以下SQL语句:
\begin{listing}[H]
	\caption{执行AVG函数语句}
	\label{code:avgfunctionclause}
\begin{minted}{mysql}
        SELECT AVG(amount) AS avg_amount FROM payment;      
\end{minted}
\end{listing}

输出的结果为:

\begin{listing}[H]
	\caption{执行AVG函数语句的结果}
	\label{code:avgfunctionclauseresult}
\begin{minted}{mysql}
        +------------+
        | avg_amount |
        +------------+
        |   4.201356 |
        +------------+
        1 row in set (0.00 sec)
\end{minted}
\end{listing}

\begin{orangebox}[frametitle={Tips 12.2.1 AVG()函数注意事项}]
        \begin{enumerate}
                \item AVG()只能用来确定特定数值列的平均值,而且列名必须作为函数参数给出。
                \item AVG()函数忽略列值为NULL的行。
        \end{enumerate}       
\end{orangebox}


\subsection{COUNT()函数}

看以下SQL语句:
\begin{listing}[H]
	\caption{执行COUNT函数语句}
	\label{code:countfunctionclause}
\begin{minted}{mysql}
        SELECT COUNT(amount) AS count_amount FROM payment;      
\end{minted}
\end{listing}

输出的结果为:

\begin{listing}[H]
	\caption{执行COUNT函数语句的结果}
	\label{code:countfunctionclauseresult}
\begin{minted}{mysql}
        +--------------+
        | count_amount |
        +--------------+
        |        16044 |
        +--------------+
        1 row in set (0.00 sec)
\end{minted}
\end{listing}

\begin{orangebox}[frametitle={Tips 12.3.1 COUNT()函数注意事项}]
        如果指定列名,则指定列的值为空的行被COUNT()函数忽略,但如果COUNT()函数中用的是星号(*),则不忽略。
\end{orangebox}

\subsection{MAX()函数}

看以下SQL语句:
\begin{listing}[H]
	\caption{执行MAX函数语句}
	\label{code:maxfunctionclause}
\begin{minted}{mysql}
        SELECT MAX(amount) AS max_amount FROM payment;      
\end{minted}
\end{listing}

输出的结果为:

\begin{listing}[H]
	\caption{执行MAX函数语句的结果}
	\label{code:maxfunctionclauseresult}
\begin{minted}{mysql}
        +------------+
        | max_amount |
        +------------+
        |      11.99 |
        +------------+
        1 row in set (0.00 sec)
\end{minted}
\end{listing}

\begin{orangebox}[frametitle={Tips 12.3.1 MAX()函数注意事项}]
\begin{enumerate}
        \item MySQL允许将它用来返回任意列中的最大值,包括返回文本列中的最大值。
        \item MAX()函数忽略列值为NULL的行。
\end{enumerate}    
\end{orangebox}

\subsection{MIN()函数}

MIN()函数与MAX()函数类似,具体可以看以下例子:

\begin{listing}[H]
	\caption{执行MIN函数语句}
	\label{code:minfunctionclause}
\begin{minted}{mysql}
        SELECT MIN(amount) AS min_amount FROM payment;      
\end{minted}
\end{listing}

输出的结果为:

\begin{listing}[H]
	\caption{执行MIN函数语句的结果}
	\label{code:minfunctionclauseresult}
\begin{minted}{mysql}
        +------------+
        | min_amount |
        +------------+
        |       0.00 |
        +------------+
        1 row in set (0.00 sec)
\end{minted}
\end{listing}


\subsection{SUM()函数}

\begin{listing}[H]
	\caption{执行SUM函数语句}
	\label{code:sumfunctionclause}
\begin{minted}{mysql}
        SELECT SUM(amount) AS min_amount FROM payment WHERE payment_id REGEXP '^[1-7]7$';     
\end{minted}
\end{listing}

输出的结果为:

\begin{listing}[H]
	\caption{执行SUM函数语句的结果}
	\label{code:combinationaggregatefunctionclauseresult}
\begin{minted}{mysql}
        +------------+
        | min_amount |
        +------------+
        |      31.93 |
        +------------+
        1 row in set (0.01 sec)
\end{minted}
\end{listing}

\subsection{组合聚集函数}

可以把之前的语句组合起来:

\begin{listing}[H]
	\caption{组合聚集函数语句}
	\label{code:combinationaggregatefunctionclause}
\begin{minted}{mysql}
        SELECT AVG(amount) AS avg_amount,
               COUNT(amount) AS count_amount,
               MAX(amount) AS max_amount,
               MIN(amount) AS min_amount
        FROM payment;
\end{minted}
\end{listing}

输出的结果为:

\begin{listing}[H]
	\caption{组合聚集函数语句的结果}
	\label{code:sumfunctionclauseresult}
\begin{minted}{mysql}
        +------------+--------------+------------+------------+
        | avg_amount | count_amount | max_amount | min_amount |
        +------------+--------------+------------+------------+
        |   4.201356 |        16044 |      11.99 |       0.00 |
        +------------+--------------+------------+------------+
        1 row in set (0.00 sec)
\end{minted}
\end{listing}

\section{分组数据}
\subsection{创建分组}

分组是在SELECT语句的GROUP BY子句中建立的,我们可以看两个store中的inventory数量:
\begin{listing}[H]
	\caption{创建分组语句}
	\label{code:setgroupclause}
\begin{minted}{mysql}
        SELECT store_id,COUNT(inventory_id) AS num_inevntory FROM inventory GROUP BY store_id;
\end{minted}
\end{listing}

输出的结果为:

\begin{listing}[H]
	\caption{创建分组语句的结果}
	\label{code:setgroupclauseresult}
\begin{minted}{mysql}
        +----------+---------------+
        | store_id | num_inevntory |
        +----------+---------------+
        |        1 |          2270 |
        |        2 |          2311 |
        +----------+---------------+
        2 rows in set (0.00 sec)
\end{minted}
\end{listing}

\begin{orangebox}[frametitle={Tips 13.1.1 GROUP BY子句注意事项}]
        \begin{enumerate}
                \item GROUP BY子句可以包含任意数目的列,这使得能对分组进行嵌套。
                \item 在GROUP BY子句中嵌套了分组,数据将在最后规定的分组上进行汇总。
                \item GROUP BY子句中列出的每个列都必须是检索列或有效的表达式(但不能是聚集函数)。
                \item 如果在SELECT中使用表达式,则必须在GROUP BY子句中指定相同的表达式。不能使用别名。
                \item 除聚集计算语句外,SELECT语句中的每个列都必须在GROUP BY子句中给出。
                \item 分组列中具有NULL值,则NULL将作为一个分组返回。如果列中有多行NULL值,它们将分为一组。
                \item GROUP BY子句在WHERE子句之后,ORDER BY子句之前。
        \end{enumerate}    
\end{orangebox}

使用WITH ROLLUP可以得到分组汇总的值:

\begin{listing}[H]
	\caption{使用WITH ROLLUP语句}
	\label{code:usewithrillupclause}
\begin{minted}{mysql}
        SELECT store_id,COUNT(inventory_id) AS num_inevntory 
        FROM inventory 
        GROUP BY store_id 
        WITH ROLLUP;
\end{minted}
\end{listing}

输出的结果为:

\begin{listing}[H]
	\caption{使用WITH ROLLUP语句的结果}
	\label{code:usewithrillupclauseresult}
\begin{minted}{mysql}
        +----------+---------------+
        | store_id | num_inevntory |
        +----------+---------------+
        |        1 |          2270 |
        |        2 |          2311 |
        |     NULL |          4581 |
        +----------+---------------+
        3 rows in set (0.00 sec)
\end{minted}
\end{listing}

\subsection{过滤分组}

\begin{orangebox}[frametitle={Tips 13.2.1 HAVING注意事项}]
        \begin{enumerate}
                \item WHERE过滤指定的是行而不是分组。
                \item HAVING支持所有WHERE操作符。
                \item WHERE在数据分组前进行过滤,HAVING在数据分组后进行过滤。
        \end{enumerate}    
\end{orangebox}

可以看到以下的SQL语句:

\begin{listing}[H]
	\caption{使用过滤分组语句}
	\label{code:usehavingclause}
\begin{minted}{mysql}
        SELECT film_id,COUNT(inventory_id) AS num_inevntory 
        FROM inventory GROUP BY film_id 
        HAVING COUNT(inventory_id) = 7 
        And film_id REGEXP '^[1-9]9.$';
\end{minted}
\end{listing}

输出的结果为:

\begin{listing}[H]
	\caption{使用过滤分组语句的结果}
	\label{code:usehavingclauseresult}
\begin{minted}{mysql}
        +---------+---------------+
        | film_id | num_inevntory |
        +---------+---------------+
        |     391 |             7 |
        |     397 |             7 |
        |     590 |             7 |
        |     698 |             7 |
        |     790 |             7 |
        |     890 |             7 |
        |     891 |             7 |
        |     892 |             7 |
        |     895 |             7 |
        |     993 |             7 |
        +---------+---------------+
        10 rows in set (0.00 sec)
\end{minted}
\end{listing}

\subsection{分组和排序}

\begin{table}[H]
        \caption{ \textbf{ORDER BY 与 GROUP BY 的对比}}
        \centering
        \begin{tabular}{p{4cm}p{4cm}}
            \toprule
            \textbf{ORDER BY} & \textbf{GROUP BY} \\ 
            \midrule
            排序产生的输出 & 分组行,但输出可能不是分组的顺序 \\ 
            任意列都可以使用(甚至非选择的列也可以使用) & 只可能使用选择列或表达式列,而且必须使用每个选择列表达式 \\ 
            不一定需要 & 如果与聚集函数一起使用(或表达式),则必须使用 \\ 
            \bottomrule
        \end{tabular}
\end{table}
     
现在将上面那个例子用降序输出:

\begin{listing}[H]
	\caption{使用过滤分组降序语句}
	\label{code:usehavingdescclause}
\begin{minted}{mysql}
        SELECT film_id,COUNT(inventory_id) AS num_inevntory 
        FROM inventory GROUP BY film_id 
        HAVING COUNT(inventory_id) = 7 
        And film_id REGEXP '^[1-9]9.$'
        ORDER BY film_id DESC;
\end{minted}
\end{listing}

输出的结果为:

\begin{listing}[H]
	\caption{使用过滤分组语句的结果}
	\label{code:usehavingdescclauseresult}
\begin{minted}{mysql}
        +---------+---------------+
        | film_id | num_inevntory |
        +---------+---------------+
        |     993 |             7 |
        |     895 |             7 |
        |     892 |             7 |
        |     891 |             7 |
        |     890 |             7 |
        |     790 |             7 |
        |     698 |             7 |
        |     590 |             7 |
        |     397 |             7 |
        |     391 |             7 |
        +---------+---------------+
        10 rows in set (0.00 sec)
\end{minted}
\end{listing}

\subsection{SELECT子句顺序}

\begin{table}[H]
        \caption{\textbf{SQL聚集函数}}%标题
        \centering%把表居中
        \begin{tabular}{ccl}%四个c代表该表一共四列,内容全部居中
        \toprule%第一道横线
        \textbf{子句} & \textbf{说明} & \textbf{是否必须使用} \\
        \midrule
        SELECT & 要返回的列或表达式 & 是 \\
        FROM & 从中检索数据的表 & 仅在从表选择数据时使用 \\
        WHERE & 行级过滤 & 否 \\
        GROUP BY & 分组说明 & 仅在按组计算聚集时使用 \\
        HAVING & 组级过滤 & 否 \\
        ORDER BY & 输出排序顺序 & 否 \\
        LIMIT & 要检索的行数 & 否 \\
        \bottomrule
        \end{tabular}
\end{table}  

\section{使用子查询}

\subsection{利用子查询进行过滤}

\begin{redbox}[frametitle={Defination 14.1.1 查询(query)}]
        任何SQL语句都是查询。但此术语一般指SELECT语句。
\end{redbox}

\begin{redbox}[frametitle={Defination 14.1.2 子查询(subquery)}]
        嵌套在其他查询中的查询。
\end{redbox}

例如,我想知道顾客id为17、27、37、47、57、67、77、87、97的地址,但是customer表中只有地址的id,那么我们就可以采用子查询:

\begin{listing}[H]
	\caption{使用子查询的语句}
	\label{code:usesubqueryclause}
\begin{minted}{mysql}
        SELECT address 
        FROM address 
        WHERE address_id IN ( SELECT address_id 
                              FROM customer 
                              WHERE customer_id REGEXP '^[1-9]7$');
\end{minted}
\end{listing}

输出的结果为:

\begin{listing}[H]
	\caption{使用子查询的语句的结果}
	\label{code:usesubqueryclauseresult}
\begin{minted}{mysql}
        +------------------------+
        | address                |
        +------------------------+
        | 270 Toulon Boulevard   |
        | 217 Botshabelo Place   |
        | 1440 Fukuyama Loop     |
        | 686 Garland Manor      |
        | 943 Tokat Street       |
        | 1586 Guaruj Place      |
        | 1692 Ede Loop          |
        | 1740 Portoviejo Avenue |
        | 1599 Plock Drive       |
        +------------------------+
        9 rows in set (0.00 sec)
\end{minted}
\end{listing}

\begin{orangebox}[frametitle={Tips 14.1.1 子查询注意事项}]
        \begin{enumerate}
                \item 在WHERE子句中使用子查询,该保证SELECT语句具有与WHERE子句中相同数目的列。
                \item 虽然子查询一般与IN操作符结合使用,但也可以用于测试等于(=)、不等于(<>)等。
                \item 子查询不算一种特别有效的方法,尽量不要用。
        \end{enumerate}    
\end{orangebox}

\subsection{作为计算字段使用子查询}

例如,我们可以执行把每个staff经手的订单加起来:

\begin{listing}[H]
	\caption{作为计算字段使用子查询的语句}
	\label{code:asculculationusesubqueryclause}
\begin{minted}{mysql}
        SELECT staff_id,first_name,last_name,(SELECT COUNT(amount) 
                                              FROM payment 
                                              WHERE payment.staff_id = staff.staff_id ) AS sum_payment 
        FROM staff 
        ORDER BY staff_id;
\end{minted}
\end{listing}

输出的结果为:

\begin{listing}[H]
	\caption{作为计算字段使用子查询的语句的结果}
	\label{code:asculculatiousesubqueryclauseresult}
\begin{minted}{mysql}
        +----------+------------+-----------+-------------+
        | staff_id | first_name | last_name | sum_payment |
        +----------+------------+-----------+-------------+
        |        1 | Mike       | Hillyer   |        8054 |
        |        2 | Jon        | Stephens  |        7990 |
        +----------+------------+-----------+-------------+
        2 rows in set (0.01 sec)
\end{minted}
\end{listing}

\begin{redbox}[frametitle={Defination 14.2.1 相关子查询(correlated subquery)}]
        涉及外部查询的子查询。
\end{redbox}

其中子查询的WHERE语句就是相关子查询,任何时候只要列名可能有多义性,就必须使用这种语法(表名和列名由一个句点分隔)。如果不采用这种完全限定列名,那么MySQL就会把payment表中staff\_id与自身比较,从而出错:
\begin{listing}[H]
	\caption{作为计算字段错误使用子查询的语句}
	\label{code:asculculationusesubqueryfaultclause}
\begin{minted}{mysql}
        SELECT staff_id,first_name,last_name,(SELECT COUNT(amount) 
                                              FROM payment 
                                              WHERE staff_id = staff_id ) AS sum_payment 
        FROM staff 
        ORDER BY staff_id;
\end{minted}
\end{listing}

输出的结果为:

\begin{listing}[H]
	\caption{作为计算字段错误使用子查询的语句的结果}
	\label{code:asculculatiousesubqueryfaultclauseresult}
\begin{minted}{mysql}
        +----------+------------+-----------+-------------+
        | staff_id | first_name | last_name | sum_payment |
        +----------+------------+-----------+-------------+
        |        1 | Mike       | Hillyer   |       16044 |
        |        2 | Jon        | Stephens  |       16044 |
        +----------+------------+-----------+-------------+
        2 rows in set (0.01 sec)
\end{minted}
\end{listing}

\section{联结表}
\subsection{联结}

\begin{redbox}[frametitle={Defination 15.1.1 外键(foreign key)}]
        外键为某个表中的一列,它包含另一个表的主键值,定义了两个表之间的关系。
\end{redbox}

\begin{redbox}[frametitle={Defination 15.1.2 可伸缩性(scale)}]
        能够适应不断增加的工作量而不失败。设计良好的数据库或应用程序称之为可伸缩性好(scale well)。
\end{redbox}

\subsection{等值联结(equijoin)}

例如,可以联结address表和city表,输出前十个address和city的名称:

\begin{listing}[H]
        \caption{创建联结的语句}
	\label{code:createjoinclause}
\begin{minted}{mysql}
        SELECT address,city,district 
        FROM address,city 
        WHERE address.city_id = city.city_id 
        ORDER BY address,city 
        LIMIT 10;
\end{minted}
\end{listing}

输出的结果为:

\begin{listing}[H]
	\caption{创建联结的语句的结果}
	\label{code:createjoinclauseresult}
\begin{minted}{mysql}
        +--------------------------------+--------------+--------------+
        | address                        | city         | district     |
        +--------------------------------+--------------+--------------+
        | 1 Valle de Santiago Avenue     | Brindisi     | Apulia       |
        | 1001 Miyakonojo Lane           | Taizz        | Taizz        |
        | 1002 Ahmadnagar Manor          | Huixquilucan | Mxico        |
        | 1003 Qinhuangdao Street        | Purwakarta   | West Java    |
        | 1006 Santa Brbara dOeste Manor | Owo          | Ondo & Ekiti |
        | 1009 Zanzibar Lane             | Arecibo      | Arecibo      |
        | 1010 Klerksdorp Way            | Graz         | Steiermark   |
        | 1013 Tabuk Boulevard           | Kanchrapara  | West Bengali |
        | 1014 Loja Manor                | Ambattur     | Tamil Nadu   |
        | 1016 Iwakuni Street            | Kingstown    | St George    |
        +--------------------------------+--------------+--------------+
        10 rows in set (0.00 sec)
\end{minted}
\end{listing}

\begin{orangebox}[frametitle={Tips 15.2.1 、注意完全限定列名}]
        \begin{enumerate}
                \item 在引用的列可能出现二义性时,必须使用完全限定列名(用一个点分隔的表名和列名)。
                \item 在联结两个表时,你实际上做的是将第一个表中的每一行与第二个表中的每一行配对。WHERE子句作为过滤条件,它只包含那些匹配给定条件(这里是联结条件)的行。
                \item 应该保证所有联结都有WHERE子句,否则MySQL将返回比想要的数据多得多的数据。
                \item 有时我们会听到返回称为叉联结(cross join)的笛卡儿积的联结类型。
        \end{enumerate}
\end{orangebox}        
        
\begin{redbox}[frametitle={Defination 15.2.1 笛卡儿积(cartesian product)}]
        由没有联结条件的表关系返回的结果为笛卡儿积。检索出的行的数目将是第一个表中的行数乘以第二个表中的行数。
\end{redbox}

下面用staff表和address表来演示以下没有完全限定列名和有限定列名的情况:
\begin{listing}[H]
        \caption{没有使用完全限定列名的创建联结的语句}
	\label{code:faultcartesiancreatejoinclause}
\begin{minted}{mysql}
        SELECT first_name,last_name,address 
        FROM staff,address 
        ORDER BY address,first_name 
        LIMIT 30;
\end{minted}
\end{listing}

输出的结果为:

\begin{listing}[H]
	\caption{没有使用完全限定列名的创建联结的语句的结果}
	\label{code:faultcartesiancreatejoinclauseresult}
\begin{minted}{mysql}
        +------------+-----------+--------------------------------+
        | first_name | last_name | address                        |
        +------------+-----------+--------------------------------+
        | Jon        | Stephens  | 1 Valle de Santiago Avenue     |
        | Mike       | Hillyer   | 1 Valle de Santiago Avenue     |
        | Jon        | Stephens  | 1001 Miyakonojo Lane           |
        | Mike       | Hillyer   | 1001 Miyakonojo Lane           |
        | Jon        | Stephens  | 1002 Ahmadnagar Manor          |
        | Mike       | Hillyer   | 1002 Ahmadnagar Manor          |
        | Jon        | Stephens  | 1003 Qinhuangdao Street        |
        | Mike       | Hillyer   | 1003 Qinhuangdao Street        |
        | Jon        | Stephens  | 1006 Santa Brbara dOeste Manor |
        | Mike       | Hillyer   | 1006 Santa Brbara dOeste Manor |
        | Jon        | Stephens  | 1009 Zanzibar Lane             |
        | Mike       | Hillyer   | 1009 Zanzibar Lane             |
        | Jon        | Stephens  | 1010 Klerksdorp Way            |
        | Mike       | Hillyer   | 1010 Klerksdorp Way            |
        | Jon        | Stephens  | 1013 Tabuk Boulevard           |
        | Mike       | Hillyer   | 1013 Tabuk Boulevard           |
        | Jon        | Stephens  | 1014 Loja Manor                |
        | Mike       | Hillyer   | 1014 Loja Manor                |
        | Jon        | Stephens  | 1016 Iwakuni Street            |
        | Mike       | Hillyer   | 1016 Iwakuni Street            |
        | Jon        | Stephens  | 102 Chapra Drive               |
        | Mike       | Hillyer   | 102 Chapra Drive               |
        | Jon        | Stephens  | 1027 Banjul Place              |
        | Mike       | Hillyer   | 1027 Banjul Place              |
        | Jon        | Stephens  | 1027 Songkhla Manor            |
        | Mike       | Hillyer   | 1027 Songkhla Manor            |
        | Jon        | Stephens  | 1029 Dzerzinsk Manor           |
        | Mike       | Hillyer   | 1029 Dzerzinsk Manor           |
        | Jon        | Stephens  | 1031 Daugavpils Parkway        |
        | Mike       | Hillyer   | 1031 Daugavpils Parkway        |
        +------------+-----------+--------------------------------+
        30 rows in set (0.00 sec)
\end{minted}
\end{listing}

最终实际生成了1206条结果,显然staff有两条记录,我们可以查询一下address的行数:

\begin{listing}[H]
        \caption{查询staff行数的语句}
	\label{code:selectstaffrowclause}
\begin{minted}{mysql}
        SELECT COUNT(address) AS num_address FROM address;
\end{minted}
\end{listing}

输出的结果为:

\begin{listing}[H]
	\caption{查询staff行数的语句的结果}
	\label{code:selectstaffrowclauseresult}
\begin{minted}{mysql}
        +-------------+
        | num_address |
        +-------------+
        |         603 |
        +-------------+
        1 row in set (0.00 sec)
\end{minted}
\end{listing}

显然,staff有603条,那么所谓笛卡尔积就是2*603=1206条,显然是不对的,下面展示正确的SQL语句:

\begin{listing}[H]
        \caption{使用完全限定列名的创建联结的语句}
	\label{code:rightcartesiancreatejoinclause}
\begin{minted}{mysql}
        SELECT first_name,last_name,address 
        FROM staff,address 
        WHERE staff.address_id = address.address_id 
        ORDER BY first_name,last_name;
\end{minted}
\end{listing}

输出的结果为:

\begin{listing}[H]
	\caption{使用完全限定列名的语句的结果}
	\label{code:rightcartesiancreatejoinclauseresult}
\begin{minted}{mysql}
        +------------+-----------+----------------------+
        | first_name | last_name | address              |
        +------------+-----------+----------------------+
        | Jon        | Stephens  | 1411 Lillydale Drive |
        | Mike       | Hillyer   | 23 Workhaven Lane    |
        +------------+-----------+----------------------+
        2 rows in set (0.00 sec)
\end{minted}
\end{listing}

\subsection{内部联结}

等值联结(equijoin)是基于两个表之间的相等测试。这种联结也称为内部联结。因此我们可以用下面的语句返回和上面例子同样的结果:
\begin{listing}[H]
        \caption{使用内部联结的语句}
	\label{code:useinnerjoinclause}
\begin{minted}{mysql}
        SELECT first_name,last_name,address 
        FROM staff INNER JOIN address 
        ON staff.address_id = address.address_id 
        ORDER BY first_name,last_name;
\end{minted}
\end{listing}

输出的结果为:

\begin{listing}[H]
	\caption{使用内部联结的语句的结果}
	\label{code:useinnerjoinclauseresult}
\begin{minted}{mysql}
        +------------+-----------+----------------------+
        | first_name | last_name | address              |
        +------------+-----------+----------------------+
        | Jon        | Stephens  | 1411 Lillydale Drive |
        | Mike       | Hillyer   | 23 Workhaven Lane    |
        +------------+-----------+----------------------+
        2 rows in set (0.00 sec)
\end{minted}
\end{listing}

\subsection{联结多个表}

\begin{listing}[H]
        \caption{联结多个表的语句}
	\label{code:usemultijoinclause}
\begin{minted}{mysql}
        SELECT title, name 
        FROM film_category, film, category 
        WHERE film_category.film_id = film.film_id 
        AND film_category.category_id = category.category_id 
        AND film_category.film_id REGEXP '^[1-9]7$' 
        ORDER BY title, name;
        
\end{minted}
\end{listing}

输出的结果为:

\begin{listing}[H]
	\caption{使用内部联结的语句的结果}
	\label{code:usemultijoinclauseresult}
\begin{minted}{mysql}
        +-------------------+----------+
        | title             | name     |
        +-------------------+----------+
        | ALONE TRIP        | Music    |
        | ANONYMOUS HUMAN   | Sports   |
        | ARIZONA BANG      | Classics |
        | BABY HALL         | Foreign  |
        | BASIC EASY        | Travel   |
        | BERETS AGENT      | Action   |
        | BIRDS PERDITION   | New      |
        | BOONDOCK BALLROOM | Travel   |
        | BRIDE INTRIGUE    | Action   |
        +-------------------+----------+
        9 rows in set (0.00 sec)
\end{minted}
\end{listing}

\section{创建高级联结}
\subsection{使用表别名}

仍然采用上面的例子进行输出:

\begin{listing}[H]
        \caption{使用表别名的语句}
	\label{code:usetableasclause}
\begin{minted}{mysql}
        SELECT title, name 
        FROM film_category AS a, film AS b, category AS c
        WHERE a.film_id = b.film_id 
        AND a.category_id = c.category_id 
        AND a.film_id REGEXP '^[1-9]7$' 
        ORDER BY title, name;
        
\end{minted}
\end{listing}

输出的结果为:

\begin{listing}[H]
	\caption{使用表别名的语句的结果}
	\label{code:usetableasclauseresult}
\begin{minted}{mysql}
        +-------------------+----------+
        | title             | name     |
        +-------------------+----------+
        | ALONE TRIP        | Music    |
        | ANONYMOUS HUMAN   | Sports   |
        | ARIZONA BANG      | Classics |
        | BABY HALL         | Foreign  |
        | BASIC EASY        | Travel   |
        | BERETS AGENT      | Action   |
        | BIRDS PERDITION   | New      |
        | BOONDOCK BALLROOM | Travel   |
        | BRIDE INTRIGUE    | Action   |
        +-------------------+----------+
        9 rows in set (0.00 sec)
\end{minted}
\end{listing}

\begin{orangebox}[frametitle={Tips 16.1.1 注意表别名}]
        表别名只在查询执行中使用。与列别名不一样,表别名不返回到客户机。
\end{orangebox}   

\subsection{自联结}

可以看以下例子,假如id为577的customer反映有买的东西有问题,现在需要找到这个顾客买的inventory,看看其他买这些东西的顾客是否有问题:

\begin{listing}[H]
        \caption{使用自联结的语句}
	\label{code:useselfjoinclause}
\begin{minted}{mysql}
        SELECT DISTINCT r1.customer_id 
        FROM rental AS r1,rental AS r2 
        WHERE r1.inventory_id = r2.inventory_id 
        AND r2.customer_id = 577 
        ORDER BY r1.customer_id
        LIMIT 10; 
\end{minted}
\end{listing}

输出的结果为:

\begin{listing}[H]
	\caption{使用自联结的语句的结果}
	\label{code:useselfjoinclauseresult}
\begin{minted}{mysql}
        +-------------+
        | customer_id |
        +-------------+
        |          11 |
        |          14 |
        |          23 |
        |          26 |
        |          34 |
        |          37 |
        |          44 |
        |          46 |
        |          50 |
        |          51 |
        +-------------+
        10 rows in set (0.00 sec)
\end{minted}
\end{listing}

\begin{orangebox}[frametitle={Tips 16.2.1 用自联结而不用子查询 }]
        自联结虽然最终的结果与子查询语句是相同的,但有时候处理联结远比处理子查询快得多。
\end{orangebox} 

\subsection{自然联结}

对内部联结的例子重新用自然联结写:
\begin{listing}[H]
        \caption{使用自然联结的语句}
	\label{code:usenatualjoinclause}
\begin{minted}{mysql}
        SELECT s.first_name,s.last_name,a.address 
        FROM staff AS s, address AS a
        WHERE s.address_id = a.address_id 
        ORDER BY s.first_name,s.last_name;
\end{minted}
\end{listing}

输出的结果为:

\begin{listing}[H]
	\caption{使用自然联结的语句的结果}
	\label{code:usenatualjoinclauseresult}
\begin{minted}{mysql}
        +------------+-----------+----------------------+
        | first_name | last_name | address              |
        +------------+-----------+----------------------+
        | Jon        | Stephens  | 1411 Lillydale Drive |
        | Mike       | Hillyer   | 23 Workhaven Lane    |
        +------------+-----------+----------------------+
        2 rows in set (0.00 sec)
\end{minted}
\end{listing}

自然联结只能选择那些唯一的列。这一般是通过对表使用通配符(SELECT *),对所有其他表的列使用明确的子集来完成的,迄今为止我们建立的每个内部联结都是自然联结,

\subsection{外部联结}

\begin{redbox}[frametitle={Defination 16.3.1 外部联结}]
        联结包含了那些在相关表中没有关联行的行。
\end{redbox}

外部联结主要是OUTER LEFT/RIGHT JOIN,用哪边就是输出所有行的表,例如,所有的电影都是英文的,现在我要检索出所有语言,就需要运用外联结:

\begin{listing}[H]
        \caption{使用外联结的语句}
	\label{code:useouterjoinclause}
\begin{minted}{mysql}
        SELECT l.name,f.title 
        FROM language AS l LEFT OUTER JOIN film AS f 
        ON l.language_id = f.language_id 
        ORDER BY l.language_id DESC 
        LIMIT 10;
\end{minted}
\end{listing}

输出的结果为:

\begin{listing}[H]
	\caption{使用外联结的语句的结果}
	\label{code:useouterjoinclauseresult}
\begin{minted}{mysql}
        +----------+-------------------+
        | name     | title             |
        +----------+-------------------+
        | German   | NULL              |
        | French   | NULL              |
        | Mandarin | NULL              |
        | Japanese | NULL              |
        | Italian  | NULL              |
        | English  | ZOOLANDER FICTION |
        | English  | ZHIVAGO CORE      |
        | English  | YOUNG LANGUAGE    |
        | English  | YENTL IDAHO       |
        | English  | WRONG BEHAVIOR    |
        +----------+-------------------+
        10 rows in set (0.00 sec)
\end{minted}
\end{listing}

\subsection{使用带聚集函数的联结}

\begin{listing}[H]
        \caption{使用带聚集函数的语句}
	\label{code:useaggregatefunctionclause}
\begin{minted}{mysql}
        SELECT l.name,COUNT(f.film_id) AS num_film 
        FROM language AS l LEFT OUTER  JOIN film AS f 
        ON l.language_id = f.language_id 
        GROUP BY l.language_id 
        ORDER BY l.language_id;
\end{minted}
\end{listing}

输出的结果为:

\begin{listing}[H]
	\caption{使用带聚集函数的语句的结果}
	\label{code:useaggregatefunctionclauseresult}
\begin{minted}{mysql}
        +----------+----------+
        | name     | num_film |
        +----------+----------+
        | English  |     1000 |
        | Italian  |        0 |
        | Japanese |        0 |
        | Mandarin |        0 |
        | French   |        0 |
        | German   |        0 |
        +----------+----------+
        6 rows in set (0.00 sec)
\end{minted}
\end{listing}

\begin{orangebox}[frametitle={Tips 16.5.1 外联结注意事项}]
        \begin{enumerate}
                \item 没有*=操作符,MySQL不支持简化字符*=和=*的使用。
                \item 左右联结可以相互转换。
        \end{enumerate}       
\end{orangebox}

\section{组合查询}
\subsection{组合查询}

\begin{redbox}[frametitle={Defination 17.1.1 复合查询(compound query)或并(union)}]
        执行多个查询(多条SELECT语句),并将结果作为单个查询结果集返回。
\end{redbox}

\begin{orangebox}[frametitle={Tips 17.1.2 组合查询注意事项}]
        组合相同表的两个查询完成的工作与具有多个WHERE子句条件的单条查询完成的工作相同。   
\end{orangebox}
\subsection{使用UNION}

可以看以下例子,分别对address\_id和customer\_id施加条件:


\begin{listing}[H]
        \caption{使用UNION的语句}
	\label{code:useunionclause}
\begin{minted}{mysql}
        SELECT first_name,last_name 
        FROM customer 
        WHERE address_id REGEXP '^[1-5]7$' 
        UNION SELECT first_name,last_name 
        FROM customer 
        WHERE customer_id LIKE '7_' 
        ORDER BY first_name;
\end{minted}
\end{listing}

输出的结果为:

\begin{listing}[H]
	\caption{使用UNION的语句的结果}
	\label{code:useunionclauseresult}
\begin{minted}{mysql}
        +------------+-----------+
        | first_name | last_name |
        +------------+-----------+
        | ANNA       | HILL      |
        | BEVERLY    | BROOKS    |
        | CHRISTINA  | RAMIREZ   |
        | CHRISTINE  | ROBERTS   |
        | DENISE     | KELLY     |
        | HEATHER    | MORRIS    |
        | IRENE      | PRICE     |
        | JANE       | BENNETT   |
        | KAREN      | JACKSON   |
        | KATHY      | JAMES     |
        | LORI       | WOOD      |
        | RACHEL     | BARNES    |
        | SARAH      | LEWIS     |
        | TAMMY      | SANDERS   |
        | THERESA    | WATSON    |
        +------------+-----------+
        15 rows in set (0.00 sec)
\end{minted}
\end{listing}

上面的结果可以用多条WHERE子句完成:
\begin{listing}[H]
        \caption{使用WHERE复现的语句}
	\label{code:usewhererefindclause}
\begin{minted}{mysql}
        SELECT first_name,last_name 
        FROM customer 
        WHERE address_id REGEXP '^[1-5]7$' 
        OR customer_id LIKE '7_' 
        ORDER BY first_name;
\end{minted}
\end{listing}

输出的结果为:

\begin{listing}[H]
	\caption{使用WHERE复现的语句的结果}
	\label{code:usewhererefindclauseresult}
\begin{minted}{mysql}
        +------------+-----------+
        | first_name | last_name |
        +------------+-----------+
        | ANNA       | HILL      |
        | BEVERLY    | BROOKS    |
        | CHRISTINA  | RAMIREZ   |
        | CHRISTINE  | ROBERTS   |
        | DENISE     | KELLY     |
        | HEATHER    | MORRIS    |
        | IRENE      | PRICE     |
        | JANE       | BENNETT   |
        | KAREN      | JACKSON   |
        | KATHY      | JAMES     |
        | LORI       | WOOD      |
        | RACHEL     | BARNES    |
        | SARAH      | LEWIS     |
        | TAMMY      | SANDERS   |
        | THERESA    | WATSON    |
        +------------+-----------+
        15 rows in set (0.00 sec)
\end{minted}
\end{listing}

\begin{orangebox}[frametitle={Tips 17.2.1 UNION规则}]
        \begin{enumerate}
                \item UNION必须由两条或两条以上的SELECT语句组成,语句之间用关键字UNION分隔。
                \item UNION中的每个查询必须包含相同的列、表达式或聚集函数(不过各个列不需要以相同的次序列出)。
                \item 列数据类型必须兼容:类型不必完全相同,但必须是DBMS可以隐含地转换的类型。
        \end{enumerate}       
\end{orangebox}

\subsection{包含或取消重复的行}

\begin{orangebox}[frametitle={Tips 17.3.1 UNION包含或取消重复的行}]
        \begin{enumerate}
                \item UNION从查询结果集中自动去除了重复的行。
                \item 如果想返回所有匹配行,可使用UNION ALL而不是UNION。
        \end{enumerate}       
\end{orangebox}

我将上面例子改一下,用下面的语句反应出有重复行和没有重复行的区别:

\begin{listing}[H]
        \caption{使用UNION不包含重复行的语句}
	\label{code:useunionunrepeatclause}
\begin{minted}{mysql}
        SELECT first_name,last_name 
        FROM customer 
        WHERE address_id REGEXP '^[1-5]7$' 
        UNION SELECT first_name,last_name 
        FROM customer 
        WHERE customer_id LIKE '_3' 
        ORDER BY first_name;
\end{minted}
\end{listing}

输出的结果为:

\begin{listing}[H]
	\caption{使用UNION不包含重复行的语句的结果}
	\label{code:useunionunrepeatclauseresult}
\begin{minted}{mysql}
        +------------+------------+
        | first_name | last_name  |
        +------------+------------+
        | ANNA       | HILL       |
        | ASHLEY     | RICHARDSON |
        | BEVERLY    | BROOKS     |
        | CHRISTINE  | ROBERTS    |
        | HEATHER    | MORRIS     |
        | KAREN      | JACKSON    |
        | LOUISE     | JENKINS    |
        | PHYLLIS    | FOSTER     |
        | SARAH      | LEWIS      |
        +------------+------------+
        9 rows in set (0.00 sec)
\end{minted}
\end{listing}

下面是包含重复行的:

\begin{listing}[H]
        \caption{使用UNION包含重复行的语句}
	\label{code:useunionrepeatclause}
\begin{minted}{mysql}
        SELECT first_name,last_name 
        FROM customer 
        WHERE address_id REGEXP '^[1-5]7$' 
        UNION ALL SELECT first_name,last_name 
        FROM customer 
        WHERE customer_id LIKE '_3' 
        ORDER BY first_name;
\end{minted}
\end{listing}

输出的结果为:

\begin{listing}[H]
	\caption{使用UNION包含重复行的语句的结果}
	\label{code:useunionrepeatclauseresult}
\begin{minted}{mysql}
        +------------+------------+
        | first_name | last_name  |
        +------------+------------+
        | ANNA       | HILL       |
        | ANNA       | HILL       |
        | ASHLEY     | RICHARDSON |
        | BEVERLY    | BROOKS     |
        | CHRISTINE  | ROBERTS    |
        | CHRISTINE  | ROBERTS    |
        | HEATHER    | MORRIS     |
        | HEATHER    | MORRIS     |
        | KAREN      | JACKSON    |
        | KAREN      | JACKSON    |
        | LOUISE     | JENKINS    |
        | PHYLLIS    | FOSTER     |
        | SARAH      | LEWIS      |
        | SARAH      | LEWIS      |
        +------------+------------+
        14 rows in set (0.00 sec)
\end{minted}
\end{listing}

显然多了5条记录,这就是UNION ALL的作用。

\subsection{对组合查询结果排序}

\begin{orangebox}[frametitle={Tips 17.4.1 组合查询结果排序注意事项}]
        在用UNION组合查询时,只能使用一条ORDER BY子句,它必须出现在最后一条SELECT语句之后。
\end{orangebox}

本章所有的语句的例子都使用了ORDER BY,这里就不赘述举例了。

\begin{orangebox}[frametitle={Tips 17.4.2 组合不同的表}]
        使用UNION的组合查询可以应用不同的表,而不仅仅用于相同的表。
\end{orangebox}

\section{全文本搜索}
\subsection{启用全文本搜索支持}

一般在创建表时启用全文本搜索.如下所示:

\begin{listing}[H]
        \caption{启用全文本搜索支持的语句}
	\label{code:usefulltextclause}
\begin{minted}{mysql}
        CREATE TABLE productnotes
        (
                note_id int                             NOT NULL AUTO INCREMENT,
                prod_id char(10)                        NOT NULL,
                note_date datetime                      NOT NULL,
                note_text text                          NULL,
                PRIMARY KEY(note_id),
                FULLTEXT(note_text)
        )ENGINE=MYISAM:
\end{minted}
\end{listing}

\begin{orangebox}[frametitle={Tips 18.1.1 FULLTEXT注意事项}]
        \begin{enumerate}
                \item 可以在创建表时指定FULLTEXT,或者在稍后指定(在这种情况下所有已有数据必须立即索引)。
                \item 不要在导入数据时使用FULLTEXT。
        \end{enumerate}
\end{orangebox}      


\subsection{进行全文本搜索}
其中Match()指定被搜索的列,Against()指定要使用的搜索表达式。

\begin{listing}[H]
        \caption{进行全文本搜索的语句}
	\label{code:fulltextselectclause}
\begin{minted}{mysql}
        SELECT description 
        FROM film 
        WHERE MATCH(description) AGAINST('Monastery')
        ORDER BY description
        LIMIT 10; 
\end{minted}
\end{listing}


输出的结果为:

\begin{listing}[H]
	\caption{进行全文本搜索的语句的结果}
	\label{code:fulltextselectclauseresult}
\begin{minted}{mysql}
+------------------------------------------------------------------------------------------------------+
| description                                                                                          |
+------------------------------------------------------------------------------------------------------+
| A Action-Packed Yarn of a Womanizer And a Lumberjack who must Chase a Sumo Wrestler in A Monastery   |
| A Amazing Story of a Mad Cow And a Dog who must Kill a Husband in A Monastery                        |
| A Astounding Character Study of a Composer And a Student who must Overcome a Composer in A Monastery |
| A Astounding Character Study of a Womanizer And a Hunter who must Escape a Robot in A Monastery      |
| A Astounding Documentary of a Butler And a Explorer who must Challenge a Butler in A Monastery       |
| A Awe-Inspiring Story of a Feminist And a Cat who must Conquer a Dog in A Monastery                  |
| A Beautiful Character Study of a Robot And a Astronaut who must Overcome a Boat in A Monastery       |
| A Beautiful Documentary of a Astronaut And a Crocodile who must Discover a Madman in A Monastery     |
| A Brilliant Panorama of a Mad Scientist And a Mad Cow who must Meet a Pioneer in A Monastery         |
| A Fanciful Story of a Man And a Sumo Wrestler who must Outrace a Student in A Monastery              |
+------------------------------------------------------------------------------------------------------+
10 rows in set (0.00 sec)
\end{minted}
\end{listing}

\begin{orangebox}[frametitle={Tips 18.2.1 MATCH注意事项}]
        \begin{enumerate}
                \item 使用完整的Match(),说明传递给Match()的值必须与FULLTEXT()定义中的相同。
                \item 除非使用BINARY方式(本章中没有介绍),否则全文本搜索不区分大小写。
                \item (使用LIKE)以不特别有用的顺序返回数据。前者(使用全文本搜索)返回以文本匹配的良好程度排序的数据。
        \end{enumerate}
\end{orangebox}  


\begin{listing}[H]
        \caption{进行全文本搜索排序的语句}
	\label{code:fulltextselectorderclause}
\begin{minted}{mysql}
        SELECT description,
               MATCH(description) AGAINST('Monastery') AS relevance_score
        FROM film
        LIMIT 5;
\end{minted}
\end{listing}

输出的结果为(太长处理了一下,但大概意思是这样):

\begin{listing}[H]
	\caption{进行全文本搜索排序的语句的结果}
	\label{code:fulltextselectorderclauseresult}
\begin{minted}{mysql}
        +------------------------------------------+-----------------+
        | description                              | relevance_score |
        +------------------------------------------+-----------------+
        | A Epic Drama of a Feminist And a Mad Scie|               0 |
        | A Astounding Epistle of a Database Admini|               0 |
        | A Astounding Reflection of a Lumberjack A|               0 |
        |a Boy who must Redeem a Boy in A Monastery| 2.15655422210693|
        +------------------------------------------+-----------------+
        5 rows in set (0.00 sec)
\end{minted}
\end{listing}

\begin{orangebox}[frametitle={Tips 18.2.2 计算出来的等级值}]
        \begin{enumerate}
                \item 不包含词的行等级为0(因此不被前一例子中的WHERE子句选择)。
                \item 包含词的每行都有一个等级值,文本中词靠前的行的等级值比词靠后的行的等级值高。
        \end{enumerate}
\end{orangebox}  

\subsection{使用查询扩展}

采用之前的第一个例子:
\begin{listing}[H]
        \caption{使用查询扩展的语句}
	\label{code:usesearchexpansionclause}
\begin{minted}{mysql}
        SELECT description 
        FROM film 
        WHERE MATCH(description) AGAINST('Monastery' WITH QUERY EXPANSION)
        LIMIT 10; 
\end{minted}
\end{listing}


输出的结果为:

\begin{listing}[H]
	\caption{使用查询扩展的语句的结果}
	\label{code:usesearchexpansionclauseresult}
\begin{minted}{mysql}
        +-----------------------------------------------------------------------+
        | description                                                           |
        +-----------------------------------------------------------------------+
        | A Action-Packed Yarn of a Womanizer And a Lumberjack who must Chase a |
        |Sumo Wrestler in A Monastery                                           |
        | A Unbelieveable Character Study of a Cat And a Database Administrator |
        |who must Pursue a Teacher in A Monastery                               |
        | A Intrepid Reflection of a Technical Writer And a Hunter who must     |
        |Defeat a Sumo Wrestler in A Monastery                                  |
        | A Action-Packed Story of a Pioneer And a Technical Writer who must    |
        |Discover a Forensic Psychologist in An Abandoned Amusement Park        |
        | A Awe-Inspiring Drama of a Technical Writer And a Composer who must   |
        |Reach a Pastry Chef in A U-Boat                                        |
        | A Stunning Character Study of a Mad Scientist And a Mad Cow who must  | 
        |Kill a Car in A Monastery                                              |
        | A Awe-Inspiring Character Study of a Robot And a Sumo Wrestler who    |
        |must Discover a Womanizer in A Shark Tank                              |
        | A Awe-Inspiring Tale of a Forensic Psychologist And a Woman who must  |
        |Challenge a Database Administrator in Ancient Japan                    |
        | A Awe-Inspiring Reflection of a Pastry Chef And a Teacher who must    |
        |Overcome a Sumo Wrestler in A U-Boat                                   |
        | A Awe-Inspiring Documentary of a Robot And a Mad Scientist who must   |
        |Reach a Database Administrator in A Shark Tank                         |
        +-----------------------------------------------------------------------+
        10 rows in set (0.01 sec)
\end{minted}
\end{listing}

\subsection{布尔文本搜索}

\begin{orangebox}[frametitle={Tips 18.4.1 布尔文本搜索可以提供的细节}]
        \begin{enumerate}
                \item 要匹配的词。
                \item 要排斥的词(如果某行包含这个词,则不返回该行,即使它包含其他指定的词也是如此)。
                \item 排列提示(指定某些词比其他词更重要,更重要的词等级更高)。
                \item 表达式分组。
                \item 另外一些内容。
        \end{enumerate}
\end{orangebox}  

\begin{orangebox}[frametitle={Tips 18.4.2 没有FULLTEXT索引也可以使用}]
        布尔方式不同于迄今为止使用的全文本搜索语法的地方在于,即使没有定义FULLTEXT索引,也可以使用它。
\end{orangebox} 
        
\begin{listing}[H]
        \caption{使用布尔文本搜索的语句}
	\label{code:usebooleanmodeclause}
\begin{minted}{mysql}
        SELECT title FROM film WHERE MATCH(title) AGAINST('VIRGIN' IN BOOLEAN MODE);
\end{minted}
\end{listing}


输出的结果为:

\begin{listing}[H]
	\caption{使用布尔文本搜索的语句的结果}
	\label{code:usebooleanmodeclauseresult}
\begin{minted}{mysql}
        +--------------+
        | title        |
        +--------------+
        | SEA VIRGIN   |
        | VIRGIN DAISY |
        +--------------+
        2 rows in set (0.01 sec)
\end{minted}
\end{listing}

\begin{table}[H]
        \caption{\textbf{全文本布尔操作符}}%标题
        \centering%把表居中
        \begin{tabular}{ll}%四个c代表该表一共四列,内容全部居中
        \toprule%第一道横线
        布尔操作符&说明\\
        \midrule%第二道横线 
        + & 包含,词必须存在 \\
        - & 排除,词必须不出现 \\
        > & 包含,而且增加等级值 \\
        < & 包含,且减少等级值 \\
        () & 把词组成子表达式(允许这些子表达式作为一个组被包含、排除、排列等) \\
        ~ & 取消一个词的排序值 \\
        * & 词尾的通配符 \\
        "" & 定义一个短语(与单个词的列表不一样,它匹配整个短语以便包含或排除这个短语) \\
        \bottomrule%第三道横线
        \end{tabular}
\end{table} 
下面举个例子,我把上面那个例子要求不检索DAISY: 


        
\begin{listing}[H]
        \caption{使用复杂布尔文本搜索的语句}
	\label{code:usedifficultbooleanmodeclause}
\begin{minted}{mysql}
        SELECT title FROM film WHERE MATCH(title) AGAINST('VIRGIN -DAISY' IN BOOLEAN MODE);
\end{minted}
\end{listing}


输出的结果为:

\begin{listing}[H]
	\caption{使用复杂布尔文本搜索的语句的结果}
	\label{code:usedifficultbooleanmodeclauseresult}
\begin{minted}{mysql}
        +------------+
        | title      |
        +------------+
        | SEA VIRGIN |
        +------------+
        1 row in set (0.01 sec)
\end{minted}
\end{listing}

\begin{orangebox}[frametitle={Tips 18.4.3 排列而不排序}]
        布尔方式中,不按等级值降序排序返回的行。
\end{orangebox} 

\begin{orangebox}[frametitle={Tips 18.4.4 全文本搜索的使用说明}]
        \begin{enumerate}
                \item 在索引全文本数据时,短词被忽略且从索引中排除。短词定义为那些具有3个或3个以下字符的词(如果需要,这个数目可以更改)。
                \item MySQL带有一个内建的非用词(stopword)列表,这些词在索引全文本数据时总是被忽略。
                \item MySQL规定了一条50\%规则,如果一个词出现在50\%以上的行中,则将它作为一个非用词忽略。
                \item 如果表中的行数少于3行,则全文本搜索不返回结果(因为每个词或者不出现,或者至少出现在50\%的行中)。
                \item 忽略词中的单引号。
                \item 不具有词分隔符(包括日语和汉语)的语言不能恰当地返回全文本搜索结果。
        \end{enumerate}
\end{orangebox}  

\section{插入数据}
\subsection{插入完整的行}

例如:

\begin{listing}[H]
        \caption{插入完整的行的语句}
	\label{code:insertrowclause}
\begin{minted}{mysql}
        INSERT INTO customer 
        VALUES (NULL, 1, 'RED', 'BAMBOO', 'chenqw0409@163.com', 77, 1, NOW(), NOW());
\end{minted}
\end{listing}

这个语句可以换用一种更加安全的方式:

\begin{listing}[H]
        \caption{更安全的插入完整的行的语句}
	\label{code:moresafeinsertrowclause}
\begin{minted}{mysql}
        INSERT INTO customer (customer_id,
                              store_id,
                              first_name,
                              last_name,
                              email,
                              address_id,
                              active,
                              create_date,
                              last_update)
        VALUES (NULL,
                2,
                'CHEN',
                'QIWEI',
                'chenqw0621@edu.jlu.com', 
                177,
                1,
                NOW(),
                NOW());
\end{minted}
\end{listing}

\begin{orangebox}[frametitle={Tips 19.1.1 插入列的注意事项}]
        \begin{enumerate}
                \item 总是使用列的列表 ,使用列的列表能使SQL代码继续发挥作用,即使表结构发生了变化。
                \item 不管使用哪种INSERT语法,都必须给出VALUES的正确数目。
                \item 果表的定义允许,则可以在INSERT操作中省略某些列。(字段运行空值或表定义给出默认值)
                \item 如果数据检索是最重要的,则可以通过在INSERT和INTO之间添加关键字LOW\_PRIORITY,指示MySQL降低INSERT语句的优先级。
        \end{enumerate}
\end{orangebox}

\subsection{插入多个行}

可以直接采用以下例子:

\begin{listing}[H]
        \caption{插入多个行的语句}
	\label{code:insertmultirowclause}
\begin{minted}{mysql}
        INSERT INTO customer (customer_id,
                              store_id,
                              first_name,
                              last_name,
                              email,
                              address_id,
                              active,
                              create_date,
                              last_update)
        VALUES (NULL,
                2,
                'BLUE',
                'BAMBOO',
                'chenqw0621@edu.jlu.com', 
                277,
                2,
                NOW(),
                NOW()
                ),
                (NULL,
                2,
                'YELLOW',
                'BAMBOO',
                'chenqw0621@edu.jlu.com', 
                277,
                2,
                NOW(),
                NOW()
                );
\end{minted}
\end{listing}

\subsection{插入检索出的数据}

可以将另一个表中检索出来的数据插入进当前表:

\begin{listing}[H]
        \caption{插入检索出数据的语句}
	\label{code:insertselectrowclause}
\begin{minted}{mysql}
        INSERT INTO customer_copy1 (customer_id,
                                   store_id,
                                   first_name,
                                   last_name,
                                   email,
                                   address_id,
                                   active,
                                   create_date,
                                   last_update)
        SELECT customer_id,
               store_id,
               first_name,
               last_name,
               email,
               address_id,
               active,
               create_date,
               last_update
        FROM customer;
\end{minted}
\end{listing}

\begin{orangebox}[frametitle={Tips 19.3.1 INSERT SELECT中的列名在插入中的作用}]
        MySQL不关心SELECT返回的列名。它使用的是列的位置,因此SELECT中的第一列(不管其列名)将用来填充表列中指定的第一个列,这对于从使用不同列名的表中导入数据是非常有用的。
\end{orangebox} 

\section{更新和删除数据}
\subsection{更新数据}

UPDATE既可以更新表中特定行,也可以更新表中所有行,要注意WHERE子句。

例如,现在我要把first\_name为BLUE的客户的邮箱更新了,可以参考以下语句:

\begin{listing}[H]
        \caption{更新记录的语句}
	\label{code:updaterowclause}
\begin{minted}{mysql}
        UPDATE customer
        SET email = 'chen1172305218@gmail.com'
        WHERE customer_id = 603;
\end{minted}
\end{listing}

下面演示更新多个列:
\begin{listing}[H]
        \caption{更新多个记录的语句}
	\label{code:updatemultirowclause}
\begin{minted}{mysql}
        UPDATE customer
        SET email = 'cqw1172305218@163.com',
            first_name = 'GREEN'
        WHERE customer_id = 604;
\end{minted}
\end{listing}

\begin{orangebox}[frametitle={Tips 20.1.1 UPDATE的注意事项}]
        \begin{enumerate}
                \item UPDATE语句中可以使用子查询,使得能用SELECT语句检索出的数据更新列数据。
                \item 如果用UPDATE语句更新多行,并且在更新这些行中的一行或多行时出一个现错误,则整个UPDATE操作被取消(错误发生前更新的所有行被恢复到它们原来的值)。为即使是发生错误,也继续进行更新,可使用IGNORE关键字。(UPDATE IGNORE customers)
        \end{enumerate}
\end{orangebox}


删除某个列的值可以把他设为NULL:
\begin{listing}[H]
        \caption{UPDATE删除值的语句}
	\label{code:updatedeleteclause}
\begin{minted}{mysql}
        UPDATE customer
        SET email = 'NULL'
        WHERE customer_id = 604;
\end{minted}
\end{listing}

\subsection{删除数据}

DELETE不仅可以从表中删除特定的行,也可以从表中删除所有行,例如:

\begin{listing}[H]
        \caption{删除一行的语句}
	\label{code:deleterowclause}
\begin{minted}{mysql}
        DELETE FROM customer
        WHERE customer_id = 604;
\end{minted}
\end{listing}

\begin{orangebox}[frametitle={Tips 20.2.1 DELETE的注意事项}]
        \begin{enumerate}
                \item DELETE语句从表中删除行,但是DELETE不删除表本身。
                \item 如果想从表中删除所有行,不要使用DELETE,可使用TRUNCATE TABLE语句。
                \item MySQL没有撤销(undo)按钮。应该非常小心地使用UPDATE和DELETE。
        \end{enumerate}
\end{orangebox}
\end{document}